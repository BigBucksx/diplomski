\section{Modalna analiza}
Modalna analiza je postupak određivanja osnovnih dinamičkih parametara linearnog
sustava s ciljem definiranja matematičkog modela ponašanja sustava pod utjecajem
dinamičkih sila. Modalna analiza temelji se na \textit{principu superpozicije},
odnosno na činjenici da se ukupni odziv sustava može zapisati kao linearna
kombinacija odziva pojedinih modova.
\par

Razmotrimo jednadžbu gibanja sustava s više stupnjeva slobode pobuđenog proizvoljnom
silom:
\begin{equation}\label{eq:mdof_jdn_gibanja_opca_pobuda}
    \mm\vtor{u}{:} + \kk\vtor{u}{} = \vtor{p(t)}{}
\end{equation}

Klasično rješenje jednadžbe gibanja \eqref{eq:mdof_jdn_gibanja_opca_pobuda}
prikazano je u poglavlju ~\ref{mdof_prisilne} na primjeru sustava s dva stupnja 
slobode pobuđenog harmonijskom silom. Za sustave s više od dva stupnja slobode ili
za složenije sile pobude, rješavanje jednadžbe gibanja na klasični način može biti
izuzetno teško ili nemoguće. U takvim slučajevima, jednadžba gibanja se rješava
postupcima modalne analize.
\par

Iz poglavlja ~\ref{slobodne_oscilacije} znamo da je rješenje jednadžbe gibanja
slobodnog titranja sustava s više stupnjeva slobode linearna kombinacija
odziva svih pojedinih modova odnosno:
\begin{equation}\label{eq:mdf_ukupno_rjesenje}
    u(t)=\sum_{r=1}^N\mm\psi_rq_r(t)
\end{equation}

Uvrštavanjem \eqref{eq:mdf_ukupno_rjesenje} u \eqref{eq:mdof_jdn_gibanja_opca_pobuda} 
dobijemo:
\begin{equation}\label{eq:mdf_modalna_1}
    \sum_{r=1}^N\mm\psi_r\ddot{q}_r(t) + \sum_{r=1}^N\kk\psi_rq_r(t)=p(t)
\end{equation}

Množenjem jednadžbe \eqref{eq:mdf_modalna_1} s $\psi_n^T$ dobijemo:
\begin{equation}\label{eq:mdf_modalna_2}
    \sum_{r=1}^N\psi_n^T\mm\psi_r\ddot{q}_r(t)+\sum_{r=1}^N\psi_n^T\kk\psi_rq_r(t)=\psi_n^Tp(t)
\end{equation}

Zbog svojstva ortogonalnosti, isčezavaju svi članovi sumacija osim $n$-tog člana, pa
preostaje:
\begin{equation}\label{eq:mdf_modalna_jednadzba_nesredjeno}
    \psi_n^T\,\mm\,\psi_n\,\ddot{q}_n(t)\;+\;\psi_n^T\,\kk\,\psi_n\,q_r(t)=\psi_n^T\,p(t)
\end{equation}

Koristeći relacije iz \eqref{eq:modalna_masa} i \eqref{eq:modalna_krutost} jednadžba
\eqref{eq:mdf_modalna_jednadzba_nesredjeno} poprima slijedeći oblik:
\begin{equation}\label{eq:mdf_modalna_jednadzba_sredjeno}
    M_n\ddot{q}_n(t)+K_nq_n(t)=P_n(t)
\end{equation}

Gdje su $M_n$, $K_n$ i $P_n$ poopćena masa, krutost i opterećenje $n$-tog moda.

Postupkom, prikazanim u jednadžbama \eqref{eq:mdf_ukupno_rjesenje},
\eqref{eq:mdf_modalna_1}, \eqref{eq:mdf_modalna_2},
\eqref{eq:mdf_modalna_jednadzba_nesredjeno}, jednadžbu gibanja sustava predstavljenu
sustavom diferencijalnih jednadžbi sveli smo na skup međusobno neovisnih
diferencijalnih jednadžbi. Drugim riječima, sustav od $N$ stupnjeva slobode razložen
je na $N$ međusobno neovisnih podsustava s jednim stupnjem slobode (princip superpozicije), 
pri čemu $n$-ti podsustav prikazuje odziv sustava u $n$-tom modu. Podsustave 
nazivamo \textit{poopćeni sustav za n-ti mod}. "Jednadžba gibanja" 
poopćenog sustava za $n$-ti oblik titranja predstavljena je diferencijalnom jednadžbom
\eqref{eq:mdf_modalna_jednadzba_sredjeno} čije rješenje predstavlja modalnu 
koordinatu $n$-tog moda $q_n(t)$.
\par

Odziv $n$-tog moda je:
\begin{equation}\label{eq:mdf_odziv_ntog_moda}
    u_n(t)=\psi_nq_n(t)
\end{equation}

Da bismo odredili ukupni odziv sustava s $N$-stupnjeva slobode, potrebno je riješiti
$N$ modalnih jednadžbi, oblika definiranog pod \eqref{eq:mdf_modalna_jednadzba_sredjeno}.
Matrični zapis sustava modalnih jednadžbi prikazan je u nastavku.
\begin{equation}\label{eq:sustav_modalnih_jednadzbi}
    \M\vtor{q}{:} + \K\vtor{q}{}=\vtor{P(t)}{}
\end{equation}

Gdje je $\M$ matrica modalnih masa, $\K$ matrica modalnih krutosti, $\vtor{P(t)}{}$
vektor poopćenih opterećenja. Iz \eqref{eq:modalna_masa_matrica} i
\eqref{eq:modalna_krutost_matrica} znamo da su matrice $\M$ i $\K$ dijagonalne što
znači da je \eqref{eq:sustav_modalnih_jednadzbi} sustav međusobno neovisnih jednadžbi.
Rješenjem navedenog sustava, dobijemo funkcije modalnih koordinata za sve modove
sustava, a ukupni odziv definirano je linearnom kombinacijom (princip superpozicije) 
odziva svakog pojedinog moda. Odziv pojedinog moda definiran je
\eqref{eq:mdf_odziv_ntog_moda}, a ukupni odziv je:

\begin{equation}\label{eq:mdf_modalna_ukupno_rjesenje}
    u(t)=\sum_{n=1}^Nu_n(t)=\sum_{n=1}^N\psi_nq_n(t)
\end{equation}
