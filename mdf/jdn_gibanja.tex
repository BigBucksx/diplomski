\section{Jednadžba gibanja slobodnih oscilacija}\label{slobodne_oscilacije}
Općenito, sustavi s više stupnjeva slobode modelirani su kao $N$ etažni posmični
okviri. Takavi sustavi se sastoje od $N$ koncentriranih masa, što znači da je potrebno 
pratiti $N$ različitih pomaka. Drugim riječima, jednadžba gibanja takvog sustava biti će
zadana kao sustav od N diferencijalnih jednadžbi drugog reda.
\par

Sustav s više stupnjeva slobode, koji će biti razmatran u ovom radu, je dvoetažni
posmični okvir prikazan na slijedećoj slici, a osnovni pojmovi biti će objašnjeni
pomoću slobodnih oscilacija navedenog modela.

\par
ubaci sliku sustava i ekvivalentnog modela
\par

Sustavi sa slike imaju dva dinamička stupnja slobode jer su moguće dvije translacije
dviju različitih masa, pa jednadžbu gibanja opisuje sustav od dvije diferencijelne
jednadžbe drugog reda.
\begin{equation}\label{eq:sustav_diferencijalnih}
    \begin{dcases}
        m\ddot{u}_1 + (k_1+k_2)u_1 -k_2u_2 = 0\\
        m\ddot{u}_2 - k_2u_1 + k_2 u_2 = 0
    \end{dcases}
\end{equation}

Zapisano u matričnoj formi:
\begin{equation}\label{eq:sustav_diferencijalnih_matricno}
    \begin{bmatrix}
        m_1 & 0 \\
        0   & m_2
    \end{bmatrix}
    \begin{Bmatrix}
        \ddot{u}_1\\
        \ddot{u}_2
    \end{Bmatrix}
    +
    \begin{bmatrix}
        k_1+k_2 & -k_2\\
        -k_2 & k_2
    \end{bmatrix}
    \begin{Bmatrix}
        u_1\\
        u_2
    \end{Bmatrix}
    =
    \begin{Bmatrix}
        0\\
        0
    \end{Bmatrix}
\end{equation}

Odnosno
\begin{equation}\label{eq:sustav_diferencijalnih_matricni_kratko}
    \mm\vtor{u}{:}+\kk\vtor{u}{}=\vtor{0}{}
\end{equation}
Iz \eqref{eq:sustav_diferencijalnih} i \eqref{eq:sustav_diferencijalnih_matricno}
vidi se da je sustav diferencijalnih jednadžbi povezan preko krutosti odnosno
matrice krutosti. Opći oblik rješenja sustava je slijedeći:
\begin{equation}\label{eq:rjesenje_opcenitog_sustava}
    \vtor{u(t)}{} = \vtor{\psi}{}q(t)
\end{equation}

Vektor $\psi$ možemo shvatiti kao konstantu integracije, a funkcija $q(t)$ je
jednostavna harmonijska funkcija oblika:
\begin{equation}\label{eq:jednostavna_harmonijska_funkcija}
    q(t)=A\cos(\omega t) + B\sin(\omega t)
\end{equation}
Druga derivacija \eqref{eq:jednostavna_harmonijska_funkcija} jest:
\begin{equation}\label{eq:jednostavna_harmonijska_funkcija_dd}
    \ddot{q}(t)=-\omega^2(
    \underbrace{
        A\cos(\omega t) + B\sin(\omega t)
    }_{\text{$q(t)$}}
    )
    =-\omega^2q(t)
\end{equation}

Stoga, druga derivacija od \eqref{eq:rjesenje_opcenitog_sustava} glasi:
\begin{equation}\label{eq:rjesenje_opcenitog_sustava_dd}
    \vtor{u}{:}=-\omega^2q(t)\vtor{\psi}{}
\end{equation}

Uvrštavanjem \eqref{eq:rjesenje_opcenitog_sustava} i \eqref{eq:rjesenje_opcenitog_sustava_dd}
u \eqref{eq:sustav_diferencijalnih_matricni_kratko} dobijemo:
\begin{equation}
    (\vtor{\psi}{}\kk-\omega^2\vtor{\psi}{}\mm)q(t)=\vtor{0}{}
\end{equation}
Prvo trivijalno rješenje je za $q(t)=0$ što implicira da je $u(t)=0$ (sustav
miruje). Netrivijalno rješenje se dobije izjednačavanjem zagrade s nulom:

\begin{equation}\label{eq:vlastite_vrijednosti}
    \kk\{\psi\}=\omega^2\mm\{\psi\}
\end{equation}
Izraz \eqref{eq:vlastite_vrijednosti} predstavlja realni problem vlastitih
vrijednosti odnosno matrični problem vlastitih vrijednosti. Potrebno je odrediti
dvije nepoznanice: 
\begin{enumerate}
    \item vlastite vektore $\psi$
    \item vlastite skalare $\omega^2$
\end{enumerate}

Prebacivanjem nepoznanica na jednu stranu dobijemo
\begin{equation}\label{eq:vlastite_vrijednosti_homogeno}
    (\kk-\omega^2\mm)\vtor{\psi}{}=\vtor{0}{}
\end{equation}

U općenitom slučaju, izraz \eqref{eq:vlastite_vrijednosti_homogeno} predstavlja
sustav od N algebarskih jednadžbi s N nepoznanica. Trivijalno rješenje sustava je za
$\{\psi\}={0}$, a netrivijalno se određuje raspisom determinante matrice
$[[k]-\omega^2[m]]$. Raspisom determinante navedene matrice, dobije se polinom N-tog
stupnja kojeg nazivamo \textit{karakterističnim polinomom}. Nultočke polinoma
predstavljaju vlastite vrijednosti $\omega^2$, odnosno kvadrirane prirodne
frekvencije. Da bi nultočke polinoma bile realne pozitivne vrijednosti, matrice $[m]$
i $[k]$ moraju biti simetrične i pozitivno definitne. Uvijeti za pozitivnu
definitnost u građevinarstvu su slijedeći:
\begin{enumerate}
    \item za matricu $\kk$ - broj i raspored ležajeva u ispravnoj mreži mora biti
        takav da se spriječe pomaci krutog tijela
    \item za matricu $\mm$ - moraju se ukloniti stupnjevi slobode bez pridružene
        koncentrirane mase. Uklanjanje nepotrebnih stupnjeva slobode, vrši se
        statičkom kondenzacijom.
\end{enumerate}

Vlastiti vektori $\psi$ se određuju uvrštavanjem vrijednosti $\omega^2$ u matricu
$[k-\omega^2m]$, stoga je očito da vektori $\psi$ nisu jedinstveni jer i njihovi
višektratnici zadovoljavaju jednadžbu \eqref{eq:vlastite_vrijednosti_homogeno}.
Vektori $\psi$ nazivaju se \textit{modalnim vektorima}, a definiraju oblik titranja
sustava na frekvenciji $\omega$. Prvi modalni vektor $\psi_1$, naziva se temeljnim
(osnovnim) modom, a frekvencija na kojoj sustav titra u navedenom modu ($\omega_1$)
naziva se \textit{vlastitom frekvencijom temeljnog moda}.
\par

Ako su sve prirodne frekvencije različite od nula i međusobno različite, tada su svi
vlastiti vektori linearno nezavisni. Skup od n linearno nezavisnih vektora čini bazu
n-dimenzionalnog vektorskog prostora, pa je ukupno rješenje sustava diferencijalnih
jednadžbi linearna kombinacija svih pojedinačnih rješenja.
\begin{equation}\label{eq:opce_rjesenje_sustava}
    \vtor{u(t)}{}=\sum_{n=1}^N\vtor{\psi}{}_nq_n 
\end{equation}
Pri čemu je $q_n$:
\begin{equation}
    q_n=A_n\cos(\omega_n t) + B_n\sin(\omega_n t)
\end{equation}

Raspisivanjem \eqref{eq:opce_rjesenje_sustava} dobivamo:
\[
	\begin{Bmatrix}
		u_1\\
		u_2\\
		\vdots\\
		u_n
	\end{Bmatrix}
	=
	q_1(t)
	\begin{Bmatrix}
		\psi_{1,1}\\
		\psi_{2,1}\\
		\vdots\\
		\psi_{N,1}
	\end{Bmatrix}
	+
	q_2(t)
	\begin{Bmatrix}
		\psi_{1,2}\\
		\psi_{2,2}\\
		\vdots\\
		\psi_{N,2}
	\end{Bmatrix}
		+
		\cdots
		+
	q_N(t)
	\begin{Bmatrix}
		\psi_{1,N}\\
		\psi_{2,N}\\
		\vdots\\
		\psi_{N,N}
	\end{Bmatrix}
	\]
\[
	=
	\begin{bmatrix}
		q_1(t) \psi_{1,1} + q_2(t) \psi_{1,2} + q_N(t) \psi_{1,N} \\
		q_1(t) \psi_{2,1} + q_2(t) \psi_{2,2} + q_N(t) \psi_{2,N} \\
		\vdots \\
		q_1(t) \psi_{N,1} + q_2(t) \psi_{N,2} + q_N(t) \psi_{N,N}
	\end{bmatrix}
	=
	\underbrace{
	\begin{bmatrix}
		\psi_{1,1} & \psi_{1,2} & \cdots & \psi_{1_N} \\
		\psi_{2,1} & \psi_{2,2} & \cdots & \psi_{2_N} \\
		\vdots & \vdots & \ddots & \vdots \\
		\psi_{N,1} & \psi_{N,2} & \cdots & \psi_{N,N} 
	\end{bmatrix}
	}_{\text{\large{$\ppsi$}}}
	\underbrace{
	\begin{Bmatrix}
		q_1(t) \\
		q_2(t) \\
		\vdots \\
		q_n(t)
	\end{Bmatrix}
	}_{\text{\large{$\overrightarrow{q}$}}}
\]

Matricu $\ppsi$ nazivamo modalna matrica, a komponente vektora $q$ nazivaju se
modalne koordinate. Opće rješenje pod \eqref{eq:opce_rjesenje_sustava} sada možemo 
zapisati matrično kao:
\begin{equation}
    \vtor{u(t)}{} = \ppsi \vtor{q}{}
\end{equation}


Osim modalne matrice postoji i spektralna matrica ($N$x$N$) koja se sastoji od N
svojstvenih vrijednosti $\omega^2$.
\[
	\oomega^2 
	= 
	\begin{bmatrix}
		\omega_1^2 & 0 & 0 & \cdots & 0 \\
		0 & \omega_2^2 & 0 & \cdots & 0 \\
		0 & 0 & \omega_3^2 & \cdots & 0 \\
		\vdots  & \vdots  & \vdots  & \ddots &  0 \\
		0 & 0 & 0 & \cdots &  \omega_N^2 
	\end{bmatrix}
\]


Za slučaj sustava s dva stupnja slobode, definiranog sustavom diferencijalnih
jednadžbi pod \eqref{eq:sustav_diferencijalnih_matricno}, prirodne frekvencije 
$\omega_1^2$ i $\omega_2^2$ dobivene su rješavanjem kvadratne jednadžbe karakterističnog
polinoma za $\omega^2$. Modalne vektore možemo zapisati kao:
\begin{align}
    \overrightarrow{\psi}_1=
    \begin{Bmatrix}
        \psi_1\\
        \psi_2
    \end{Bmatrix}
    &=
    \begin{Bmatrix}
        \ffrac{k_1+k_2-\omega_1^2m_1}{k_2}\\
        1
    \end{Bmatrix}\\
    \overrightarrow{\psi}_2=
    \begin{Bmatrix}
        \psi_1\\
        \psi_2
    \end{Bmatrix}
    &=
    \begin{Bmatrix}
        \ffrac{k_1+k_2-\omega_1^2m_1}{k_2}\\
        1
    \end{Bmatrix}
\end{align}

Ukupno rješenje sustava jest linearna kombinacija slijedećih vektora:
\begin{equation}
    \begin{dcases}
        \vtor{u}_1(t) = \vtor{\psi}_1 q_1(t) = \vtor{\psi}_1 (A_1\cos(\omega_1 t) + B_1\sin(\omega_1 t))\\
        \vtor{u}_2(t) = \vtor{\psi}_2 q_2(t) = \vtor{\psi}_2 (A_2\cos(\omega_2 t) + B_2\sin(\omega_2 t))
    \end{dcases}
\end{equation}

Stoga, ukupno opće rješenje glasi:
\begin{equation}\label{eq:ukupno_opce_rjesenje_2dof}
    \vtor{u}(t)=\vtor{u}_1(t)+\vtor{u}_2(t)
\end{equation}

Bitno je za napomenuti da modalni vektor $\psi$ ne određuje maksimalne iznose
ordinata već samo njihov relativni odnos, tj. modalni oblik ili oblik titranja. Da
bismo dobili amplitude $A_n$ i $B_n$, potrebno je rješiti inicijalni problem oblika 
$\vtor{u(0)}{} = \vtor{u}{}$ i $\vtor{u(0)}{.}=\vtor{u}{}$. Za slučaj slobodnog titranja,
konstante $A_n$ i $B_n$ glase:
\begin{align}
    A_n&=u_n(0)\\
    B_n&=\frac{\dot{u}_n(0)}{\omega_n}
\end{align}

Shematski prikaz modova sustava s dva stupnja slobode prikazan je na slijedećoj
slici.
