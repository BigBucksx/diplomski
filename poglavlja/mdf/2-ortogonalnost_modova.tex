\section{Ortogonalnost modova}
Kao što je već pokazano, modove (modalne oblike) definiraju vektori. Dva vektora su
međusobno ortogonalna (okomita) ukoliko je njihov skalarni produkt jednak nuli.
Razmotrimo li $r$-ti i $n$-ti mod sustava, dobijemo slijedeći sustav jednadžbi 
(iz \eqref{eq:sustav_diferencijalnih_matricni_kratko}):
\begin{equation}\label{eq:pocetni_sustav_ortogonalnost}
    \begin{dcases}
        (\kk-\omega_r^2\mm)\{\psi\}_r=\{0\}\\
        (\kk-\omega_n^2\mm)\{\psi\}_n=\{0\}
    \end{dcases}
\end{equation}

Donju jednadžbu pomnožimo s $\{\psi\}_r^T$. U gornjoj jednadžbi prvo
transponiramo $\{\psi\}_r$ te ju pomnožimo s $\{\psi\}_n$. Sustav jednadžbi
\eqref{eq:pocetni_sustav_ortogonalnost} postaje:
\begin{equation}\label{eq:konacni_sustav_ortogonalnost}
    \begin{dcases}
        \{\psi\}_r^T(\kk-\omega_r^2\mm)\{\psi\}_n=0\\
        \{\psi\}_r^T(\kk-\omega_n^2\mm)\{\psi\}_n=0
    \end{dcases}
\end{equation}

Oduzimanjem gornje i donje jednadžbe dobijemo:
\begin{equation}
    (\omega_n^2-\omega_r^2)\{\psi\}_r^T\mm\{\psi\}_s=0
\end{equation}

Za $\omega_n\neq\omega_r$ vrijedi:
\begin{equation}\label{eq:ortogonalnost_masa}
    \{\psi\}_r^T\mm\{\psi\}_n=0
\end{equation}

Uvrštavanjem $\{\psi\}_r^T\mm\{\psi\}_n=0$ u bilo koju od jednadžbi iz
\eqref{eq:konacni_sustav_ortogonalnost} dobijemo:
\begin{equation}\label{eq:ortogonalnost_krutost}
    \{\psi\}_r^T\kk\{\psi\}_n=0
\end{equation}

Jednadžbe pod \eqref{eq:ortogonalnost_masa} i \eqref{eq:ortogonalnost_krutost}
govore da su modovi, pomnoženi težinskim koeficijentima iz matrice krutosti ili
matrice masa, međusobno ortogonalni (valjda). Kažemo da su modovi međusobno ortogonalni s
obzirom na matricu mase ili matricu krutosti (valjda).
\par

Poslijedica ortogonalnosti je dijagonalnost slijedećih pravokutnih matrica:
\begin{alignat}{2}
    &\text{Modalna krutost}\quad & \mathbf{K}&=\ppsi^T\kk\ppsi\label{eq:modalna_krutost_matrica}\\
    &\text{Modalna masa}\quad &\mathbf{M}&=\ppsi^T\mm\ppsi\label{eq:modalna_masa_matrica}
\end{alignat}

Članovi matrica računaju se prema slijedećim formulama:
\begin{alignat}{2}
    &\text{Za modalnu krutost}\quad & K_{n,n}&=\{\psi\}_n^Tk\{\psi\}_n\label{eq:modalna_krutost}\\
    &\text{Za modalnu masu}\quad &M_{n,n}&=\{\psi\}_n^Tk\{\psi\}_n\label{eq:modalna_masa}
\end{alignat}

Između elemenata matrica vrijedi slijedeći odnos:
\begin{equation}
    \omega_n^2=\frac{K_n}{M_n}
\end{equation}
U matričnoj formi:
\begin{equation}
    \oomega^2=\mathbf{K}\mathbf{M}^{-1}
\end{equation}

