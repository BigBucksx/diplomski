\section{Model sustava s jednim stupnjem slobode}
Modeli sustava s jednim stupnjem slobode su posmični okviri, a možemo ih shvatiti
kao idealizaciju jednokatnice. Posmični okviri se sastoje od štapova, koncentrirane
mase i viskoznog prigušivača, pri čemu je ukupna krutost sustava sadržana u krutosti
štapova, ukupna masa sustava u koncentriranoj masi i ukupno prigušenje u viskoznom
prigušivaču. Nadalje, vrijedi pretpostavaka o vertikalnoj i horizontalnoj
nestišljivosti štapova, tj. štapovi su nepromjenjive dužine.
\par
\begin{figure}[H]
    \begin{subfigure}[b]{0.5\textwidth}\label{fig:priguseni-sustav-sdf}
        \centering
        \begin{tikzpicture}
%štapovi
	\draw[thick] (0,0) -- (0,2.2)
		node[pos=0.5,left]{\textbf{\large{$k$}}};
	\draw[thick] (0,2.2) -- (4,2.2);
	\draw[thick] (4,0) -- (4,2.2);
	
%masa
	\filldraw[color=black, fill=gray] (2,2.2) circle (0.5);
	\node[draw=none, fill=none]  at(2, 3)
		{\textbf{\large{$m$}}};

%viskozni prigusivac
	\draw[thick] (0, 0) -- (1.5, 1.1);
	\draw[thick] (1.5, 1.1) -- (1.9, 1.1);
	\draw[thick] (2.1, 1.1) -- (2.5, 1.1);
	\draw[thick] (2.5, 1.1) -- (4, 2.2);

	\draw[thick] (1.9, 0.7) -- (1.9, 1.4);
	\draw[thick] (1.9, 1.4) -- (2.25, 1.4);
	\draw[thick] (1.9, 0.7) -- (2.25, 0.7)
		node[pos=0.5, below]{\textbf{\large{$c$}}};

	\draw[thick] (2.1, 1.3) -- (2.1, 0.8);


%podloga
	\draw[white, pattern=north east lines, pattern color=black] (-0.5, 0) 
	rectangle (0.5, -0.5);
	\draw[thick] (-0.5, 0) -- (0.5, 0);

	\draw[white, pattern=north east lines, pattern color=black] (3.5, 0)
	rectangle (4.5, -0.5);
	\draw[thick] (3.5, 0) -- (4.5, 0);
\end{tikzpicture}

        \caption{Posmični okvir s prigušenjem}
    \end{subfigure}
    \hfill
    \begin{subfigure}[b]{0.5\textwidth}\label{fig:priguseni-ekvivalentni-sustav-sdf}
        \centering
        \begin{tikzpicture}
	%podloga
	\draw[white, pattern=north east lines, pattern color=black] (0, 0)
	rectangle (-0.5, 2.2);
	\draw[thick] (0,0) -- (0, 2.2);

	\draw[white, pattern=north east lines, pattern color=black] (0, 0) 
	rectangle (5, -0.5);
	\draw[thick] (0,0) -- (5,0);
	
	\draw[white, pattern=north east lines, pattern color=black] (0, 0)
	rectangle (-0.5, -0.5);

	%uteg
	\filldraw[fill=gray] (1.5, 2) rectangle (3.5, 0.25);

	%kotaci
	\filldraw[fill=gray] (2, 0.125) circle (0.125);
	\filldraw[fill=gray] (3, 0.125) circle (0.125);

	%opruga
	\draw[thick, decoration={aspect=0.1, segment length=2mm,amplitude=3mm,coil},decorate] (0,1.65) -- (1.5, 1.65);

	%prigusivac
	\draw[thick] (0, 0.65) -- (0.685, 0.65);
	\draw[thick] (0.84, 0.65) -- (1.5, 0.65);

	\draw[thick] (0.685, 0.35) -- (0.685, 0.95);
	\draw[thick] (0.685, 0.95) -- (0.9, 0.95);
	\draw[thick] (0.685, 0.35) -- (0.9, 0.35);

	\draw[thick] (0.84, 0.5) -- (0.84, 0.8);

        \node[draw=none, fill=none] at (0.75, 2.2) {\textbf{\large{$k$}}}; 
        \node[draw=none, fill=none] at (2.5,   2.2) {\textbf{\large{$m$}}};
        \node[draw=none, fill=none] at (0.75, 1.1) {\textbf{\large{$c$}}};

\end{tikzpicture}

        \caption{Ekvivalentni model s prigušenjem}
    \end{subfigure}
    \vfill
    \begin{subfigure}[b]{0.5\textwidth}\label{fig:nepriguseni-sustav-sdf}
        \centering
        \begin{tikzpicture}
%štapovi
	\draw[thick] (0,0) -- (0,2.2)
		node[pos=0.5,left]{\textbf{\large{$k$}}};
	\draw[thick] (0,2.2) -- (4,2.2);
	\draw[thick] (4,0) -- (4,2.2);
	
%masa
	\filldraw[color=black, fill=gray] (2,2.2) circle (0.5);
	\node[draw=none, fill=none] at(2,3) 
	{\textbf{\large{$m$}}};

%podloga
	\draw[white, pattern=north east lines, pattern color=black] (-0.5, 0) 
	rectangle (0.5, -0.5);
	\draw[thick] (-0.5, 0) -- (0.5, 0);

	\draw[white, pattern=north east lines, pattern color=black] (3.5, 0)
	rectangle (4.5, -0.5);
	\draw[thick] (3.5, 0) -- (4.5, 0);
\end{tikzpicture}

        \caption{Posmični okvir bez prigušenja}
    \end{subfigure}
    \hfill
    \begin{subfigure}[b]{0.5\textwidth}\label{fig:nepriguseni-ekvivalentni-sustav-sdf}
        \centering
        \begin{tikzpicture}
	%podloga
	\draw[white, pattern=north east lines, pattern color=black] (0, 0)
	rectangle (-0.5, 2.2);
	\draw[thick] (0,0) -- (0, 2.2);

	\draw[white, pattern=north east lines, pattern color=black] (0, 0) 
	rectangle (5, -0.5);
	\draw[thick] (0,0) -- (5,0);
	
	\draw[white, pattern=north east lines, pattern color=black] (0, 0)
	rectangle (-0.5, -0.5);

	%uteg
	\filldraw[fill=gray] (1.5, 2) rectangle (3.5, 0.25);

	%kotaci
	\filldraw[fill=gray] (2, 0.125) circle (0.125);
	\filldraw[fill=gray] (3, 0.125) circle (0.125);

	%opruga
	\draw[thick, decoration={aspect=0.3, segment length=2mm,amplitude=3mm,coil},decorate] (0,1) -- (1.5, 1);

	\node[draw=none, fill=none] at (0.75, 1.75) {\textbf{\large{$k$}}}; 
	\node[draw=none, fill=none] at (2.5,   2.2) {\textbf{\large{$m$}}};

\end{tikzpicture}

        \caption{Ekvivalentni model bez prigušenja}
    \end{subfigure}
    \caption{Idealizirani sustav s jednim stupnjem slobode}
\end{figure}

Primjetimo da navedeni sustav s pogleda statike predstavlja ravninski problem s tri
stupnja slobode (dvije rotacije u zglobovima te translacija), stoga ga je potrebno
svesti na sustav s jednim stupnjem slobode za potrebe dinamičkog proračuna. Drugim
riječima, potrebno je ukupnu krutost sustava izraziti kao lateralnu krutost.
Postupak svođenja sustava s tri stupnja slobode na sustav s jednim stupnjem slobode
naziva se \textit{statička kondenzacija} koju u ovom radu nećemo razmatrati.
\newpage

