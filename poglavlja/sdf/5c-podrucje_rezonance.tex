\subsection{Analiza područja rezonance}
Osim rezonantne frekvencije potrebno je odrediti i pojas polovice snage vrha
frekvencijske funkcije odziva, koji je prikazan na slici \ref{fig:hpb}. Pojas polovice
snage vrha frekvencijske funkcije odziva bitan je iz dva razloga:
\begin{enumerate}
    \item osim pobude rezonantnom frekvencijom, opasne su i pobude frekvencijama iz
        njezinog okoliša. Navedeni okoliš definiran je pojasom polovice snage.
    \item Zbog praktične primjene - pojas polovice snage koristi se u pokusima
        za određivanje stupnja prigušenja konstrukcija
\end{enumerate}

Dinamički faktor pomaka $R_d$ za odziv dvostruko manje snage od odziva maksimalnog
dinamičkog faktora računa se prema slijedećoj relaciji:
\begin{equation}\label{eq:hpb}
    R_d = \frac{1}{\sqrt{2}}R_d^{max}
\end{equation}

Sa slike je vidljivo da je dinamički faktor $R_d$ iz \eqref{eq:hpb} definiran za
dvije vrijednosti frekvencije pobude: $\omega_a$ i $\omega_b$. Raspisivanjem
jednadžbe pod \eqref{eq:hpb} dobijemo:

\begin{equation}\label{eq:hpb_izvod_1}
        \frac{1}{\sqrt{(1-(\omega/\omega_n)^2)^2+(2\zeta\omega/\omega_n)^2}}=
            \frac{1}{\sqrt{2}}\frac{1}{2\zeta\sqrt{1-\zeta^2}}\\
\end{equation}

Kvadriranjem \eqref{eq:hpb_izvod_1} dobijemo:
\begin{equation}\label{eq:hpb_izvod_2}
    \left(1-\left(\frac{\omega}{\omega_n}\right)^2\right)^2
    +\left(2\zeta\frac{\omega}{\omega_n}\right)^2 =
    8\zeta^2(1-\zeta^2)
\end{equation}

Raspisivanjem i grupiranjem po $\omega/\omega_n$ dobijemo:
\begin{equation}\label{eq:hpb_izvod_3}
    \left(\frac{\omega}{\omega_n}\right)^4
    -2(1-2\zeta^2)\left(\frac{\omega}{\omega_n}\right)^2
    +1-8\zeta^2(1-\zeta^2)=0
\end{equation}

Izraz \eqref{eq:hpb_izvod_3} je kvadratna jednadžba, a njezinim rješavanjem (po
$\omega/\omega_n$) dobijemo:
\begin{equation}\label{eq:hpb_kvadratna_1}
    \left(\frac{\omega}{\omega_n}\right)^2 = 
        (1-2\zeta^2)\pm 2\zeta\sqrt{1-\zeta^2}
\end{equation}

Za $\zeta^2 \approx 0$ izraz \eqref{eq:hpb_kvadratna_1} postaje:
\begin{equation}
    \left(\frac{\omega}{\omega_n}\right)^2 \approx
        1 \pm 2\zeta
\end{equation}

Odnosno:
\begin{equation}\label{eq:hpb_kvadratna_2}
    \left(\frac{\omega}{\omega_n}\right)\approx
        \sqrt{1 \pm 2\zeta}
\end{equation}
Jednadžba \eqref{hpb_kvadratna_2}, nakon aproksimacije korijena s prva dva člana Taylorovog
reda glasi:
\begin{equation}\label{eq:hpb_kvadratna_konacno}
    \frac{\omega}{\omega_n} = 1 \pm \zeta
\end{equation}

Frekvencije $\omega_a$ i $\omega_b$ dobiju se iz \eqref{eq:hpb_kvadratna_konacno}:
\begin{align}
    \omega_a &= (1-\zeta)\omega_n\\
    \omega_b &= (1+\zeta)\omega_n\\
\end{align}

Oduzimanjem $\omega_b-\omega_a$ dobijemo:
\begin{equation}\label{eq:hpb_zeta_omega}
        2\zeta = \frac{\omega_b-\omega_a}{\omega_n}
\end{equation}

Te konačno, dijeljenjem brojnika i nazivnika s $2\pi$:
\begin{equation}\label{eq:hpb_zeta_f}
    \zeta\approx\frac{f_b-f_a}{2f_n}
\end{equation}

Gdje je $f=\omega/2\pi$ kružna frekvencija. Jednadžbe pod \eqref{eq:hpb_zeta_omega}
i \eqref{eq:hpb_zeta_f} bitne su jer omogućuju određivanje koeficijenta relativnog prigušenja
$\zeta$ bez potrebe za poznavanjem intenziteta sile pobude. 


