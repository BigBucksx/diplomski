\section{Rezonanca}
Rezonanca je pojava koja se javlja prilikom pobude rezonantnom frekvencijom.
Rezonantna frekvencija je ona frekvencija pobude za koju će dinamički koeficijent
odziva biti maksimalan. 
\par

\subsection{Rezonanca sustava s prigušenjem}
Razmatranjem krivulja frekvencijskih funkcija odziva sa slike 
\ref{fig:rd-rv-ra}, možemo uočiti da jedino maksimumi krivulja $R_v$ padaju na 
vertikalni pravac $\omega/\omega_n=1$. Isto tako, vidljivo je da maksimalni $R_d$
pada ulijevo od navedenog pravca a maksimalni $R_a$ udesno. To znači da će se 
rezonantne frekvencije dinamičkih faktora ($R_d$, $R_v$ i $R_a$) međusobno razlikovati 
te da će rezonantne frekvencije dinamičkih faktora $R_d$ i $R_a$ biti različite od
prirodne frekvencije sustava.

Rezonantne frekvencije za pojedini spektar možemo odrediti deriviranjem frekvencijske 
funkcije odziva po $\omega/\omega_n$ te izjednačavanjem prve derivacije s
nulom.
\[
    R_d = \frac{1}{\sqrt{(1-(\omega/\omega_n)^2) + (2\zeta(\omega/\omega_n))^2}}
\]
Uvodi se supstitucija $x=\omega/\omega_n$
\[
    R_d = \frac{1}{\sqrt{(1-x^2)+(2\zeta x)^2}}
\]
Deriviranjem po x dobijemo:
\[
    \frac{dR_d}{dx}=-\frac{2x^3+(4\zeta^2-2)x}{\sqrt{((1-x^2)^2+(2\zeta x)^2)^3}}
\]
Lokalni ekstrem (maksimum) dobijemo izjednačavanjem prve derivacije s nulom,
odnosno:
\[
    -\frac{2x^3+(4\zeta^2-2)x}{\sqrt{((1-x^2)^2+(2\zeta x)^2)^3}} = 0
\]
Da bi razlomak bio jednak nula izraz u brojniku mora biti jednak nula.
Izjednačavanjem brojnika s nulom dobijemo slijedeću jednadžbu:
\[
    \begin{aligned}
        2x^3+(4\zeta^2-2)x &= 0\\
        x^2 &= 1-2\zeta^2\\
        x &= \sqrt{1-2\zeta^2} \quad \text{ Uz } \zeta < \frac{1}{\sqrt{2}}
    \end{aligned}
\]
Uvrštavanjem $\omega/\omega_n$ dobijemo izraz za rezonantnu frekvenciju pomaka:
\begin{equation}\label{eq:rezonantna_frekvencija_pomak}
    \omega=\omega_n\sqrt{1-2\zeta^2}
\end{equation}
Analogno tome dobiju se izrazi za rezonantne frekvencije brzine i ubrzanja.
\begin{align}
    \text{Rezonantna frekvencija pomaka}\quad & \omega = \omega_n\sqrt{1-\zeta^2}\label{eq:rd_rezonanca}\\
    \text{Rezonantna frekvencija brzine}\quad & \omega = \omega_n\label{eq:rv_rezonanca}\\
    \text{Rezonantna frekvencija ubrzanja}\quad & \omega = \frac{\omega_n}{\sqrt{1-2\zeta^2}}\label{eq:ra_rezonanca}
\end{align}

Zbog jednostavnosti, karakteristike odziva prigušenog sustava s jednim stupnjem
slobode u rezonanci, razmotriti ćemo za slučaj $\omega = \omega_n$ uz
homogene početne uvjete. Iako se rezonantna frekvencija pomaka ne podudara s 
prirodnom frekvencijom sustava, navedene vrijednosti su vrlo bliske za mali $\zeta$. 
Vremenska funkcija pomaka prigušenog sustava prikazana jednadžbom \eqref{eq:inverz_treci_korak}

Uz homogene početne uvjete i za $\omega=\omega_n$, koeficijenti $A,B,C,\text{ i }D$
glase:
\begin{align}
    C &= 0 \label{eq:rez_koef_C}\\
    D &= -\frac{(u_{st})_0}{2\zeta}\label{eq:rez_koef_D}\\
    A &= -D = \frac{(u_{st})_0}{2\zeta}\label{eq:rez_koef_A}\\
    B &= -\frac{D\sigma}{\omega_D}-\frac{C\omega}{\omega_D}=
          \frac{(u_{st})_0}{2\sqrt{1-\zeta^2}}\label{eq:rez_koef_B}
\end{align}
Uvrštavanjem \eqref{eq:rez_koef_A}, \eqref{eq:rez_koef_B}, \eqref{eq:rez_koef_C} i \eqref{eq:koef_D}
u \eqref{eq:inverz_treci_korak} te sređivanjem dobijemo:
\begin{equation}\label{eq:rezonancaPrigušenoPomak}
    u(t)=\frac{(u_{st})_0}{2\zeta}\left[
        e^{-\sigma t} \left(
            \cos(\omega_D t)+\frac{\zeta}{\sqrt{1-\zeta^2}}\sin(\omega_D t)
            \right)
        -\cos(\omega_n t)
        \right]
\end{equation}
Iz navedene jednadžbe proizlazi da je amplituda odziva ograničena na vrijednost
(uočimo da je odziv sustava kontroliran prigušenjem):
\begin{equation}\label{eq:rezonanca_amplituda}
    u_0=\frac{(u_{st})_0}{2\zeta}
\end{equation}
Za malu vrijednost $\zeta$, vrijedi $\omega_n\approx\omega_D$ te je član
$\sin(\omega_Dt) \approx 0$. Stoga jednadžba pod \eqref{eq:rezonancaPrigušenoPomak}
postaje:
\begin{equation}\label{eq:rezonancaOdzivAproksimacija}
    u(t)\simeq\underbrace{
        \frac{(u_{st})_0}{2\zeta}[(e^{-\sigma t}-1)
        }_{\text{Krivulja ovojnice}}
        \cos(\omega_nt)]
\end{equation}

Iz jednadžbe \eqref{eq:rezonancaOdzivAproksimacija} može se zaključiti da i za
slučaj rezonance postoji prolazni i prisilni dio odziva, a ukupni odziv je razlika
između prolaznog i prisilnog djela. 
Prolazni dio odziva opisan je jednadžbom
\[
    e^{-\sigma t} \cos(\omega_n t)
\]
a prisilni:
\[
    \cos(\omega_n t)
\]
Za $t=0$ prolazni dio je maksimalan te je ukupni odziv jednak nuli. U ovisnosti o
vremenu, prolazni dio se smanjuje eksponencijalno prema zakonu $e^{-\sigma t}$. Kako
se prolazni dio smanjuje a prisilni ostaje isti ($(u_{st})_0/2\zeta$) tako raste
njihova razlika. Rastom razlike dolazi do rasta amplitude odziva, te isčezavanjem
prolaznog dijela preostaje samo prisilni te se dostiže maksimalna amplituda koja je
jednaka ($(u_{st})_0/2\zeta$). Bitno je za naglasiti da u teoretskom modelu
prolazni dio odziva isčezava tek za $t=\infty$, tj. asimptotski se približava nuli,
ali u realnosti prolazni dio odziva je zanemariv nakon određenog vremena. Shematski
prikaz dostizanja prisilnog stanja prikazan je na slijedećoj slici:

O prigušenju u rezonanci ovise slijedeći parametari odziva:
\begin{itemize}
    \item brzina dostizanje ustaljenog stanja (maksimalne amplitude) - brzina dostizanja 
        ustaljenog stanja raste proporcionalno s prigušenjem (veće prigušenje $\to$ strmija krivulja
        ovojnice $\to$ brže dostizanje ustaljenog stanja).

    \item vrijednost maksimalne amplitude - obrnuto proporcionalna od vrijednosti
        prigušenja, definirana izrazom \eqref{eq:rezonanca_amplituda}. 
        (veće prigušenje, manja maksimalna amplituda, vidljivo i u frekvencijskim funkcijama odziva)
\end{itemize}

Određivanje broja titraja koji je potreban za dostizanje ustaljenog stanja vrši
se pomoću funkcije koja opisuje krivulju ovojnice. Pretpostavka je da ekstrem
nastupa nakon $j$ titraja ($j$ je prirodni broj), a vrijeme nastupa minimuma je
$t=2\pi j/\omega$. 
\begin{equation}\label{eq:prirastEkstremaPriguseno}
    u\left(\frac{2\pi j}{\omega_n}\right) \approx
        u_0(e^{-\zeta\omega_n\frac{2\pi j}{\omega_n}}-1)\cos\left(\omega_n\frac{2\pi
        j}{\omega_n}\right)
\end{equation}
Gdje je:
\begin{table}[H]
    \begin{tabular}{c c}
        $j$ & redni broj titraja\\
        $u_0=(u_{st})_0/2\zeta$ & maksimalna amplituda\\
    \end{tabular}
\end{table}
Kako se radi o ekstremnoj vrijednosti, funkcija kosinus iznosi $\pm 1$ pa jednadžba
pod \eqref{eq:prirastEkstremaPriguseno} glasi:
\begin{equation}\label{eq:MiniMaxPriguseno}
     u\left(\frac{2\pi j}{\omega_n}\right) = u_j = 
        \pm u_0(e^{-2\pi\zeta j}-1)
\end{equation}

Za maksimume jednadžba pod \eqref{eq:MiniMaxPriguseno} postaje
\begin{equation}
    |u_j| = -u_0(e^{-2\pi\zeta j} -1) = u_0(1-e^{-2\pi\zeta j})
\end{equation}

Za relativne vrijednosti\footnote{Postotci od maksimalne amplitude}:
\begin{equation}
    u[j] = \frac{|u_j|}{u_0}=1-e^{-2j\zeta\pi}
\end{equation}
Izraz ima smisla samo za diskretne vrijednosti argumenta $j$, odnosno za $j \in
\mathbb{N}$.

\par
\begin{table}[H]
    \begin{tabular}{c | c}
       \hline
        $\zeta$ & $j$\\
        \hline
        $0.01$ & $48$\\
        \hline
        $0.02$ & $24$\\
        \hline
        $0.05$ & $10$\\
        \hline
        $0.1$ & $5$\\
        \hline
        $0.2$ & $3$\\
        \hline
    \end{tabular}
    \caption{Očitanje s grafa}
    \label{table:prirast-rezonanca-priguseno}
\end{table}

Uočimo da je uz slabije prigušenje potrebno više titraja za dostizanje ustaljenog
stanja odnosno maksimalne amplitude. Očitanja vrijednosti sa grafa prikazana su u
\ref{table:prirast-rezonanca-priguseno}.


