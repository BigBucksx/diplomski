\subsection{Dinamički koeficijenti odziva}
U prethodnim poglavljima, prisilni dio odziva je prikazan kao vremenska funkcija pomaka
pri čemu je njegova amplituda jednaka statičkom pomaku skaliranom dinamičkim
koeficijentom pomaka $R_d$. Osim vremenskom funkcijom pomaka, odziv sustava potrebno
je opisati i vremenskim funkcijama brzine i ubrzanja.

Deriviranjem vremenske funkcije pomaka po vremenu dobijemo vremensku funkciju brzine:
    \begin{align}
        \frac{\dot{u}(t)}{(u_{st})_0} &= 
            \omega R_d\cos(\omega t - \phi)\frac{\omega_n}{\omega_n}\notag\\
        \frac{\dot{u}(t)}{\omega_n p_0/k} &= 
            \frac{\omega}{\omega_n}R_d\cos(\omega t-\phi)\label{eq:derivacijaFaktorPomaka}\\
        \frac{\dot{u}(t)}{p_0/\sqrt{km}} &=
            R_v\cos(\omega t-\phi)\label{eq:dinamickiFaktorBrzine}
    \end{align}
Gdje je:
\begin{table}[H]
    \begin{tabular}{r l}
        $R_v$ & Dinamički koeficijent brzine\\
    \end{tabular}
\end{table}

Iz jednadžbi pod \eqref{eq:derivacijaFaktorPomaka} i \eqref{eq:dinamickiFaktorBrzine}
slijedi relacija:
\begin{equation}\label{eq:R_v}
    \frac{\dot{u}(0)}{p_0/\sqrt{km}}=R_v=\left(\frac{\omega}{\omega_n}\right)R_d
\end{equation}

Analogno tome, dobije se i dinamički faktor ubrzanja koji glasi:
\begin{equation}\label{eq:R_a}
    \frac{\ddot{u}(t)}{p_0/m}=-R_a=-\frac{\omega}{\omega_n}R_v
        =-\left(\frac{\omega}{\omega_n}\right)^2R_d
\end{equation}

Gdje je:
\begin{table}[H]
    \begin{tabular}{r l}
        $R_a$ & Dinamički koeficijent ubrzanja\\
    \end{tabular}
\end{table}


%Sa slike \ref{fig:rd-rv-ra} vidljivo je da prigušenje na sva tri koeficijenta najviše
%utječe u okolini točke $\omega/\omega_n=1$. Isto tako vidljivo je da je maksimum
%dinamičkog faktora pomaka pomaknut u lijevo od vertikalnog pravca koji prolazi
%točkom $\omega/\omega_n=1$, a maksimum dinamičkog faktora ubrzanja u desno. 
%Maksimum dinamičkog faktora brzine pada na pravac $\omega/\omega_n=1$. Poslijedica
%toga je da se rezonantne frekvencije sustava na pomak, brzinu i ubrzanje razlikuju.
%Rezonantne frekvencije biti će razmatrane detaljnije u slijedećim poglavljima.

Iz \eqref{eq:R_v} i \eqref{eq:R_a} slijedi da su dinamički koeficijenti pomaka,
brzine i ubrzanja u odnosu. Navedeni odnos je prikazan je slijedećom jednadžbom.
\begin{equation}\label{eq:R_d-R_v-R_a}
    \frac{R_a}{\omega/\omega_n}=R_v=\frac{\omega}{\omega_n}R_d 
\end{equation}
Iz navedene jednadžbe slijedi da je poznavanjem jedne od veličina $R_d, R_v 
\text{ ili } R_a$ moguće dobiti preostale dvije. Zbog toga što postoji odnos
između dinamičkih koeficijenata pomaka, brzine i ubrzanja opisan jednadžbom 
\eqref{eq:R_d-R_v-R_a}, moguće je sve tri navedene veličine prikazati u jednom
grafu.

Takav graf se sastoji od četiri logaritamske skale:
\begin{enumerate}
    \item horizontalne logaritamske skale koja prikazuje omjer frekvencije pobude i 
        prirodne frekvencije sustava 
    \item vertikalne logaritamske skale koja prikazuje \textit{dinamički koeficijent brzine
        $R_d$}
    \item modificirane logaritamske skale nagnute pod kutem od $+45\degree$ koja prikazuje
        \textit{dinamički koeficijent ubrzanja $R_a$}
    \item modificirane logaritamske skale nagnute pod kutem od $-45\degree$ koja prikazuje
        \textit{dinamički koeficijent pomaka $R_d$}
\end{enumerate}


