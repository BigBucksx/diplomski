\section{Model sustava s jednim stupnjem slobode}
Modeli sustava s jednim stupnjem slobode su posmični okviri, a možemo ih shvatiti
kao idealizaciju jednokatnice. Posmični okviri se sastoje od štapova, koncentrirane
mase i viskoznog prigušivača, pri čemu je ukupna krutost sustava sadržana u krutosti
štapova, ukupna masa sustava u koncentriranoj masi i ukupno prigušenje u viskoznom
prigušivaču. Nadalje, vrijedi pretpostavaka o vertikalnoj i horizontalnoj
nestišljivosti štapova, tj. štapovi su nepromjenjive dužine.


Primjetimo da navedeni sustav s pogleda statike predstavlja ravninski problem s tri
stupnja slobode (dvije rotacije u zglobovima te translacija), stoga ga je potrebno
svesti na sustav s jednim stupnjem slobode za potrebe dinamičkog proračuna. Drugim
riječima, potrebno je ukupnu krutost sustava izraziti kao lateralnu krutost.
Postupak svođenja sustava s tri stupnja slobode na sustav s jednim stupnjem slobode
naziva se \textit{statička kondenzacija} koju u ovom radu nećemo razmatrati.

\section{Jednadžba gibanja modela pri sinusnoj pobudi}
Dijagram sila za sustav s prigušenjem na koji djeluje sila pobude $p(t)$ prikazan je
na slijedećoj slici:


Prema drugom Newtonovom aksiomu vrijedi:
\begin{align}
    \Sigma F = ma = m\ddot{u}\notag\\
        p(t) - F_S - F_D = m\ddot{u} \notag\\
        m\ddot{u} + F_S + F_D = p(t) \label{eq:newton}
\end{align}

Gdje je:\\
\begin{table}[H]
\begin{tabular}{c l}
	$F_S$ & unutarnja sila \\
	$F_D$ & sila prigušenja \\
	$m$   & masa \\
        $a,\ddot{u}$   & ubrzanje\\
\end{tabular}
\end{table}

Unutarnju silu možemo zapisati kao:
\begin{equation}
	F_S = k \cdot u \label{eq:hooke}
\end{equation}

A silu prigušenja kao:
\begin{equation}
	F_D = c \cdot v = c \cdot \dot{u} \label{eq:prigusenje}
\end{equation}

Gdje je:\\
\begin{table}[H]
\begin{tabular}{c l}
	$k$ & krutost \\
	$c$ & koeficijent viskoznog prigušenja \\
	$u$ & pomak \\
	$\dot{u},v$ & brzina \\
\end{tabular}
\end{table}

Uvrštavanjem \eqref{eq:hooke} i \eqref{eq:prigusenje} u \eqref{eq:newton} dobijemo:
\begin{equation}
	m\ddot{u} + c\dot{u} + ku = p(t) \label{eq:jednadzba_opcenita_pobuda}
\end{equation}

Sila pobude $p(t)$ je sinusna sila $p_0sin(\omega t)$, pri čemu je $p_0$ amplituda, a
$\omega$ frekvencija, stoga jednadžba pod \eqref{eq:jednadzba_opcenita_pobuda}
postaje:
\begin{equation}
	m\ddot{u} + c\dot{u} + ku = p_0sin(\omega t)
\label{eq:jednadzba_sinusna_pobuda}
\end{equation} 


Jednadžba pod \eqref{eq:jednadzba_sinusna_pobuda} jest nehomogena linearna
diferencijalna jednadžba drugog reda, a riještiti ćemo ju primjenom Laplaceove
transformacije\footnote{Iako postupkom dugotrajnija (u ovom
slučaju), metoda Laplaceove transformacije nam pruža jedinstven uvid u međudjelovanje sustava
i pobude.}.
\par

Transformiranjem jednadžbe \eqref{eq:jednadzba_sinusna_pobuda} dobijemo 
slijedeću algebarsku jednadžbu u frekvencijskoj domeni:
\begin{equation}\label{eq:transformat_diferencijalna}
        m(s^2U(s)-su(0)-\dot{u}(0))+
	c(sU(s)-u(0))+
	kU(s) = 
        P(s)
\end{equation}
Gdje je $P(s)$ transformat funkcije $p(t)=p_0\sin(\omega t)$.
Sređivanjem \eqref{eq:transformat_diferencijalna} dobijemo: 
\begin{equation}\label{eq:transformat_diferencijalna_sredjeno}
    U(s)\left(ms^2+cs+k\right)-msu(0)-m\dot{u}-cu(0) = P(s)
\end{equation}

Jednadžbu \eqref{eq:transformat_diferencijalna_sredjeno} možemo rastaviti na više logičnih
cijelina:
\begin{alignat}{2}
    &\text{Dinamička krutost} & Z(s)&=ms^2+cs+k\label{eq:din_krutost}\\
    &\text{Prijenosna funkcija sustava}\quad & H(s)&=\frac{1}{Z(s)}=\frac{1}{ms^2+cs+k}\label{eq:prijenosna}\\
    &\text{Pobuda početnim uvjetima}\quad & W(s)&=(ms+c)u(0)+m\dot{u}\label{eq:pobuda_pocetni}\\
    &\text{Pobuda sinusnom silom} & P(s)&=p_0\frac{\omega}{s^2+\omega^2}\label{eq:pobuda_sinusna}
\end{alignat}

Uvrštavanjem \eqref{eq:prijenosna},\eqref{eq:pobuda_pocetnim_uvjetima} u 
\eqref{eq:transformat_diferencijalna_sredjeno} dobijemo:
\begin{equation*}
	\frac{U(s)}{H(s)}=W(s)+P(s)
\end{equation*}

Množenjem prijenosnom funkcijom sustava ($H(s)$) dobijemo:
\begin{equation}
	U(s)=\underbrace{H(s)W(s)}_{\substack{\text{konvolucija u}\\\text{vremenskoj
	domeni}}}
	+ 
	\underbrace{H(s)P(s)}_{\substack{\text{konvolucija u}\\\text{vremenskoj domeni}}}
\end{equation}

Pri čemu je:
\begin{table}[H]
\begin{tabular}{c l}
	$H(s)\cdot W(s)$ & odziv na pobudu početnim uvjetima u frekvencijskoj domeni\\
	$H(s)\cdot P(s)$ & odziv na pobudu sinusnom silom u frekvencijskoj domeni\\
\end{tabular}
\end{table}

Da bi bilo lakše naći inverze odziva u frekvencijskoj domeni potrebno je funkciju  
$H(s)$ prikazati u tabličnom obliku\footnote{postupak svođenja prijenosne funkcije
sustava na tablični oblik prikazan je u slijedećem pogavlju}.

\begin{equation}\label{eq:pfs_tablicni_oblik}
    H(s) = \frac{\omega_n^2}{\omega_D}
           \frac{1}{k}
           \frac{\omega_D}{(s+\sigma)^2+\omega_D^2}
\end{equation}

Gdje je:\\
\begin{table}[H]
    \begin{tabular}{r l}
        $\sigma=\zeta\omega_n$ & stupanj prigušenja\\
        $\omega_D=\omega_n\sqrt{1-\zeta^2}$ & vlastita frekvencija prigušenog titranja\\
        $\omega_n=\sqrt{k/m}$ & prirodna frekvencija oscilatora\\
        $\zeta=c/c_{kr}$ & koeficijent relativnog prigušenja\\
        $c_{kr}=2m\omega_n$ & kritično prigušenje
    \end{tabular}
\end{table}

Pobuda početnim uvjetima zadana je preko jednadžbe:
\begin{equation}
    \begin{split}
        H(s)W(s)=\frac{u_0}{\omega_D}&\left(
        \frac{s+\sigma}{(s+\sigma)^2+\omega_D^2} +
	\sigma\frac{\omega_D}{(s+\sigma)^2+\omega_D^2}\right)\\
        &+ \frac{\dot{u}(0)}{\omega_D}\frac{\omega_D}{(s+\sigma)^2+\omega_D}
    \end{split}
\end{equation}

Odziv u vremenskoj domeni se određuje pronalaskom inversa $\ltr$ transformacije
funkcije odziva u frekvencijskoj domeni.
\begin{equation}
	u(t)=\ltr^{-1}\{U(s)\}=\ltr^{-1}\{W(s)H(s)\}+\frac{p_0}{k}\ltr^{-1}\{H(s)P(s)\}
	\label{eq:inverz_prvi_korak}
\end{equation}

Funkcija $H(s)W(s)$ svedena je na tablični oblik stoga je moguće izravno naći inverz
Laplaceove transformacije koji glasi:
\begin{equation}
	\ltr^{-1}\{H(s)W(s)\} = e^{-\sigma t}\left[
		u(0)cos(\omega_Dt)+\left(
			\sigma\frac{u(0)}{\omega_D}+\frac{\dot{u}(0)}{\omega_D}
			\right)sin(\omega_Dt)\right] \label{eq:pobuda_pocetnim_uvjetima}
\end{equation}

Funkciju $H(s)P(s)$ nije moguće svesti na tablični oblik, pa je potrebno odrediti
konvoluciju funkcija $h(t)$ i $p(t)$. Funkcija funkcija $H(s)$ odgovara funkciji 
$h(t)$ u vremenskoj domeni, a funkcija $P(s)$ funkciji $p(t)$, stoga:
\begin{align}
        h(t)&=\ltr^{-1}\{H(s)\}=\frac{1}{k}\frac{\omega_n^2}{\omega}e^{-\sigma t}sin(\omega_Dt)\label{eq:pfs_vremenska_domena}\\
	p(t)&=\ltr^{-1}\{P(s)\}=p_0\sin(\omega t) \label{eq:pobuda_vremenska_domena}
\end{align}

Konvoluciju funkcija određujemo preko konvolucijskog integrala:
\begin{align}\label{eq:konvolucijski_integral}
	(h*p)(t)&=\int_0^th(\tau)p(t-\tau)d\tau \notag\\
		&=\frac{p_0}{k}\frac{\omega_n^2}{\omega}
		\int_0^te^{-\sigma\tau}sin(\tau)sin(t-\tau)d\tau
\end{align}

Ukupni odziv je suma odziva na pobudu početnim uvjetima i odziva na pobudu
harmonijskom silom odnosno:
\begin{equation}
    \begin{split}
    u(t)&=\underbrace{e^{-\sigma t}\left[
		u(0)cos(\omega_Dt)+\left(
                        \sigma\frac{u(0)}{\omega_D}+\frac{\dot{u}(0)}{\omega_D}
                        \right)sin(\omega_Dt) \right]}_{\text{\textbf{pobuda
                    početnim uvjetima (prolazni dio odziva)}}}\\
                    &+\underbrace{
                        \frac{p_0}{k}
                        \frac{\omega_n^2}{\omega_D}
                        \int_0^te^{-\sigma\tau}\sin(\tau)\sin(t-\tau)d\tau
                    }_{\text{\textbf{pobuda sinusnom silom}}}
    \end{split}
    \label{eq:inverz_drugi_korak}
\end{equation}

Postupak određivanja i sređivanja rješenja konvolucijskog integrala iz
\eqref{eq:konvolucijski_integral} je dugotrajan pa je u nastavku prikazano samo konačno 
rješenje.
\begin{equation}
	\begin{split}
	(p*h)(t)&=\underbrace{
			Csin(\omega t)+Dcos(\omega t)
		}_{\text{\textbf{prisilni dio odziva}}}\\
		&-
		\underbrace{
			e^{-\sigma t}
				\left(
				Dcos(\omega_D t) +
					\left(
						\frac{D\sigma}{\omega_D}+\frac{C\omega}{\omega_D}
					\right)
				sin(\omega_Dt)
				\right)
		}_{\text{\textbf{prolazni dio odziva}}}
	\end{split}
    \label{eq:konvolucijski_integral_rjesenje}
\end{equation}

Gdje je:
\begin{align}
    C &= \frac{p_0}{k}\frac{1-(\omega/\omega_n)^2}
            {[1-(\omega/\omega_n)^2]^2+[2\zeta\omega/\omega_n]^2}\label{eq:koef_C}\\
    D &= \frac{p_0}{k}\frac{-2\zeta\omega/\omega_n}
            {[1-(\omega/\omega_n)^2]^2+[2\zeta\omega/\omega_n]^2]}\label{eq:koef_D}
\end{align}

Iz \eqref{eq:konvolucijski_integral_rjesenje} i \eqref{eq:inverz_drugi_korak} možemo
zaključiti da će se prolazni dio odziva pojaviti i u slučaju homogenih i u slučaju
nehomogenih početnih uvjeta (prisutan i kod pobude početnim uvjetima i kod pobude
silom), a ovisi o karakteristikama sustava,
te se smanjuje eksponencijalno u ovisnosti o vremenu. Nakon
što prolazni dio odziva isčezne preostaje samo prisilni dio odziva, a pojavljuje se
neovisno o početnim uvjetima. Karakteristike prisilnog dijela odziva ponajviše ovise
o frekvenciji i amplitudi pobude, a zatim i o karakteristikama sustava (o prigušenju
te prirodnoj frekvenciji sustava). Karakteristike prisilnog dijela odziva detaljnije će se
razmatrati u slijedećim poglavljima.

Uvrštavanjem \eqref{eq:konvolucijski_integral_rjesenje} u
\eqref{eq:inverz_drugi_korak}, te sređivanjem dobijemo ukupno rješenje
diferencijalne jednadžbe \eqref{eq:jednadzba_sinusna_pobuda} koje glasi:
\begin{equation}\label{eq:inverz_treci_korak}
	u(t)=\underbrace{
		e^{-\sigma t}(Acos(\omega_D t)+Bsin(\omega_Dt)
		}_{\text{\textbf{prolazni dio odziva}}}
		+
	     \underbrace{
		Csin(\omega t) + Dcos(\omega t)
		     }_{\text{\textbf{prisilni dio odziva}}}
\end{equation}

Gdje je:
\begin{align}
    A &= u(0)-D\label{eq:koef_A}\\
    B &= \frac{u(0)\sigma}{\omega_D}+
         \frac{\dot{u}(0)}{\omega_D}-
         \frac{D\sigma}{\omega_D}-
         \frac{C\omega}{\omega_D}\label{eq:koef_B}
\end{align}

Graf odziva prigušenog sustava s jednim stupnjem slobode za homogene početne
uvjete $u(0)=0$ i $\dot{u}(0)=0$ prikazan je na slijedećoj slici.

Neprigušeni sustav možemo shvatiti kao poseban slučaj prigušenog sustava za slučaj
$c=0$ odnosno $\zeta=0$, pa diferencijalna jednadžba sustava postaje:
\begin{equation}\label{eq:jednadzba_gibanja_nepriguseni_nesredjeno}
	m\ddot{u}+ku=p(t)
\end{equation}

Rješenje jednadžbe gibanja možemo odrediti iz \eqref{eq:inverz_treci_korak}
određivanjem koeficijenata $A$, $B$, $C$ i $D$ za $\zeta = 0$.
\begin{align}
    A &= u(0) \label{eq:np_koef_A}\\
    B &= \frac{\dot{u}(0)}{\omega_n}-\frac{\omega/\omega_n}{1-(\omega/\omega_n)^2}\label{eq:np_koef_B}\\
    C &= \frac{p_0}{k}\frac{1}{1-(\omega/\omega_n)^2}\label{eq:np_koef_C}\\
    D &= 0\label{eq:np_koef_D}\\
    \sigma &= \omega_n\zeta=\omega_n\cdot 0=0\label{eq:np_sigma}\\
    \omega_D &= \omega_n\sqrt{1-\zeta^2}=\omega_n\sqrt{1-0}=\omega_n\label{eq:np_omega_D}
\end{align}

Uvrštavanjem \eqref{eq:np_koef_A}, \eqref{eq:np_koef_B}, \eqref{eq:np_koef_C},
\eqref{eq:np_koef_D}, \eqref{eq:np_sigma} i \eqref{eq:np_omega_D} u  \eqref{eq:inverz_treci_korak}
dobijemo rješenje diferencijalne jednadžbe pod
\eqref{eq:jednadzba_gibanja_nepriguseni_nesredjeno} 
koje glasi:
\begin{equation}
	u(t)=\underbrace{
            u(0)cos(\omega_n t)
	\left[
		\frac{\dot{u}(0)}{\omega_n}-\frac{\omega/\omega_n}{1-(\omega/\omega_n)^2}
        \right]sin(\omega_n t)\\
	}_{\text{\textbf{prolazni dio odziva}}}
        +
	\underbrace{
		\frac{1}{1-(\omega/\omega_n)^2} sin(\omega t)
	}_{\text{\textbf{prisilni dio odziva}}}
\end{equation}



\section{Statički pomak}
Zanemarivanjem ubrzanja u diferencijalnoj jednadžbi pod
\eqref{eq:jednadzba_gibanja_nepriguseni_nesredjeno} dobijemo:
\begin{equation}\label{eq:vremenska_funkcija_statickog_pomaka}
	u(t)=\frac{p_0}{k}sin(\omega t)
\end{equation}

Navedena jednadžba predstavlja \textit{vremensku funkciju statičkog pomaka}. Pomak
nazivamo statičkim jer se zanemaruje dinamički utjecaj sile pobude (pretpostavlja se
spora promjena opterećenja).
Vremenska funkcija statičkog pomaka, preko Hookeovog zakona, stavlja u odnos silu 
pobude ($p_0sin(\omega t)$) i pomak sustava $u(t)$. Zanemarivanjem funkcije sinus, dobijemo
amplitudu statičkog pomaka koja je definirana slijedećom jednadžbom:
\begin{equation}
	(u_{st})_0 = \frac{p_0}{k}
\end{equation}

\section{Frekvencijske funkcije odziva}
\subsection{Izvod}
Ponašanje sustava opisano je diferencijalnom jednadžbom drugog reda (u vremenskoj
domeni) koja nakon $\ltr$-transformacija postaje algebarska jednadžba u $s$ domeni.
Transformat odziva na sinusnu silu uz homogene početne uvjete glasi:
\begin{equation}
    U(s) = P(s)\cdot H(s) 
\end{equation}

Funkcija $H(s)$ naziva se prijenosnom funkcijom sustava, a definirana je kao
kvocjent odziva i pobude u $s$ domeni.
\begin{equation} \label{eq:PrijenosnaFunkcijaSustava}
    H(s)=\frac{U(s)}{P(s)} = \frac{1}{ms^2+cs+k}%\frac{\omega_n^2}{s^2+2\zeta\omega_ns+\omega_n^2}
\end{equation}

Prijenosna funkcija sustava obično sadrži dvije skupine karakterističnih točaka:
\begin{enumerate}
    \item polovi - nultočke nazivniku
    \item nule - nultočke brojnika
\end{enumerate}

Polovi predstavljaju točke u kojima prijenosna funkcija sustava divergira tj
($H(s)\to\infty$), a nule su točke u kojima vrijednost prijenosne funkcije iznosi
nula. Polovi i nule za polinom drugog stupnja mogu biti:
\begin{enumerate}
    \item dva različita realna broja
    \item jedan dvostruki realni broj
    \item jedan par kompleksno konjugiranih brojeva
\end{enumerate}

Uočimo da za slučaj prijenosne funkcije sustava pod
\eqref{eq:PrijenosnaFunkcijaSustava} nema nula, a polovi su jedan par kompleksno
konjugiranih brojeva (polovi su dobiveni izjednačavanjem polinoma u nazivniku s
nulom):
\begin{equation}
    p_{1,2} = -\sigma\pm\omega_Di
\end{equation}

Realni dio pola prikazuje stupanj prigušenja  a
imaginarni dio prirodnu frekvenciju prigušenog titranja. Prijenosna funkcija
zapisana preko polova glasi:
\begin{equation}
    H(s)=\frac{1}{(s-p_1)(s-p_2)}
%         \frac{\omega_n}{(s+\sigma+\omega_Di)(s+\sigma-\omega_Di)}=
%         \frac{\omega_n}{(s+\sigma)^2+\omega_D^2}
\end{equation}

Raspisivanjem dobijemo:
\begin{equation}
    H(s)=\frac{1}{(s+\sigma+\omega_Di)(s+\sigma-\omega_Di)}\frac{\omega_n^2}{k}
\end{equation}

I konačno tablični oblik dobijemo množenjem s $\omega_D/\omega_D$
\begin{equation}
    H(s)=\frac{1}{k}\frac{\omega_n^2}{\omega_D}\frac{\omega_D}{(s+\sigma^2)+\omega_D^2}
\end{equation}

Poseban slučaj prijenosne funkcije sustava, za slučaj $s=\omega i$, naziva se
\textit{frekvencijska funkcija odziva} i glasi:
\begin{equation} \label{eq:frf_nesredjeno}
    H(\omega i) = \frac{H(\omega i)}{P(\omega i)}
                = \frac{1}{-\omega^2 m + i\omega c + k}\frac{1/k}{1/k}
                = \frac{1}{k}\frac{1}{1-(\omega/\omega_n)^2+2\zeta(\omega/\omega_n)i}
\end{equation}

Član $1/k$ možemo zanemariti jer je uračunat u amplitudu statičkog pomaka $(u_{st})_0=p_0/k$,
pa je konačni oblik frekvencijske funkcije odziva:
\begin{equation}\label{eq:frf_sredjeno}
    H(\omega) = \frac{1}{\left(1-(\omega/\omega_n)^2\right)+\left(2\zeta\omega/\omega_n\right)i}
\end{equation}

Frekvencijska funkcija odziva $H(\omega)$ je funkcija argumenta $\omega$, odnosno promjenom
frekvencije pobude, mjenja se i njezina vrijednost, te je redovito je kompleksna.
Jednadžbu pod \eqref{eq:frf_sredjeno} možemo rastaviti na realni i imaginarni dio.
Rastav na realni i imaginarni dio prikazan je u nastavku.
\begin{alignat}{2}
    &\text{Realni dio} & H_r &= \frac{1-(\omega/\omega_n)^2}{(1-(\omega/\omega_n)^2)^2+(2\zeta\omega/\omega_n)^2}
        \label{eq:realni_dio_frf}\\
    &\text{Imaginarni dio}\quad & H_i &=\frac{2\zeta(\omega/\omega_n)}{(1-(\omega/\omega_n)^2)^2+(2\zeta\omega/\omega_n)^2}
        \label{eq:imaginarni_dio_frf}
\end{alignat}

Jednadžba pod \eqref{eq:frf_sredjeno} zapisana preko realnog i imaginarnog dijela
glasi:
\begin{equation}\label{eq:frf_pravokutni}
    \begin{split}
        H(\omega) &= H_r - H_ii  \\
              &= \frac{1-(\omega/\omega_n)^2}{(1-(\omega/\omega_n)^2)^2+(2\zeta\omega/\omega_n)^2}
              -\frac{2\zeta(\omega/\omega_n)}{(1-(\omega/\omega_n)^2)^2+(2\zeta\omega/\omega_n)^2}i
    \end{split}
\end{equation}

Funkciju \eqref{eq:frf_sredjeno} se može prikazati i u trigonometrijskom obliku.
Općenito, kompleksni broj se prikazuje u trigonometrijskom obliku na slijedeći
način:
\begin{equation}\label{eq:trig_zapis_kompleksni_br}
    R=|R|e^{i\phi} = |R|(\cos(\phi)+i\sin(\phi))
\end{equation}
Gdje je $|R|$ norma kompleksnog broja, a $\phi$ kut kojega kompleksni vektor zatvara
s realnom osi.
Stoga je očito da je za navedeni prikaz potrebno je odrediti magnitudu (normu) kompleksnog
vektora frekvencijske funkcije odziva te kut koji zatvara s realnom osi. Magnituda je 
zadana Pitagorinim poučkom, dakle:
\begin{equation}\label{eq:magnitudniSpektar}
    |H(\omega)|=\sqrt{(H_r^2+H_i^2)}
               =\frac{1}{\sqrt{(1-(\omega/\omega_n)^2)^2+(2\zeta(\omega/\omega_n))^2}}
\end{equation}

Kut koji kompleksni vektor zatvara s realnom osi moguće je odrediti slijedećom
jednadžbom:
\begin{equation}
        \phi(\omega)=\left|\arctan\left(\frac{2\zeta(\omega/\omega_n)}
                      {1-(\omega/\omega_n)^2}\right)\right|\label{eq:fazniSpektar}
\end{equation}

Prikazana u trigonometrijskom obliku, frekvencijsku funkciju odziva potrebno je
rastaviti na magnitudni spektar koji je definiran jednadžbom \eqref{eq:magnitudniSpektar} i
na fazni spektar koji je definiran jednadžbom \eqref{eq:fazniSpektar}.
\textbf{Napomena:} jednadžba \eqref{eq:fazniSpektar} stavljena je u apsolutnu
vrijednost zato što će se negativni predzak (kašnjenje) ili pozitivni predznak
uračunati naknadno, no više o tome u slijedećem poglavlju.

\subsection{Zapis prisilnog dijela odziva}
Prisilni dio odziva definiran definiran je jednadžbom prikazanom u nastavku:
\begin{equation}\label{eq:samo_prisilni}
    u(t)=C\sin(\omega t) + D\cos(\omega t)
\end{equation}

Prisilni odziv pod \eqref{eq:samo_prisilni} možemo zapisati u obliku $u_0\sin(\omega t
- \phi)$ 
korištenjem slijedećeg trigonometrijskog identiteta:
\begin{equation}\label{eq:prisilni_dio_odziva}
    u_0\sin(\omega t - \phi) = C\sin(\omega t) + D\cos(\omega t)
\end{equation}

Gdje je:
\begin{alignat}{2}
    &\text{Amplituda dinamičkog pomaka} & u_0 &= \sqrt{C^2+D^2}\label{eq:u_0-izvod-prvo}\\
    &\text{Kašnjenje u fazi} & \phi &= \arctan{-\frac{D}{C}}\label{eq:fi-izvod}
\end{alignat}

Raspisivanjem formule pod \eqref{eq:u_0-izvod-prvo} dobijemo:
\begin{equation}\label{eq:u_0-izvod-raspisano}
    u_0 = \frac{p_0}{k}\frac{1}{\sqrt{(1-(\omega/\omega_n)^2)^2+(2\zeta\omega/\omega_n)^2}}
        = \frac{(u_{st})_0}{\sqrt{(1-(\omega/\omega_n)^2)^2+(2\zeta\omega/\omega_n)^2}}
\end{equation}

Definiramo dinamički koeficijent pomaka ili koeficijent povećanja pomaka ($R_d$) kao omjer
amplitude dinamičkog i statičkog pomaka.
\begin{equation}\label{eq:R_d-izvod-konacno}
    R_d = \frac{u_0}{(u_{st})_0}=\frac{1}{\sqrt{(1-(\omega/\omega_n)^2)^2+(2\zeta\omega/\omega_n)^2}}
\end{equation}

Fazni kut $\phi$ dobijemo raspisivanjem izraza pod \eqref{eq:fi-izvod} te dobijemo:
\begin{equation}\label{eq:fi-izvod-konacno}
    \phi = \arctan{\frac{2\zeta(\omega/\omega_n)}{1-(\omega/\omega_n)^2}}
\end{equation}

Uvrštavanjem \eqref{eq:R_d-izvod-konacno} i \eqref{eq:fi-izvod-konacno} u
\eqref{eq:prisilni_dio_odziva} dobijemo:
\begin{equation}
    u(t) = \frac{p_0}{k}R_d\sin(\omega t - \phi) = (u_{st})_0R_d\sin(\omega t - \phi)
\end{equation}

Uočimo da je jednadžba pod \eqref{eq:R_d-izvod-konacno} ista kao i jednadžba
magnitudnog spektra frekvencijske funkcije odziva. Isto tako, uočimo i da je jednadžba
pod \eqref{eq:fi-izvod-konacno} ista kao i jednadžba faznog spektra frekvencijske
funkcije odziva. Nadalje, prisilni dio odziva je sinusoida isto kao i pobuda.
\par
 
Stoga možemo reći da frekvencijska funkcija odziva definira odnos između pobude i
odziva. Taj odnos je kompleksan jer je opisan magnitudnim i faznim spektrom
kompleksne funkcije definirane u $s$ domeni. Drugim riječima, frekvencijska funkcija
odziva nas upućuje na slijedeća svojstva (prisilnog) odziva:
\begin{enumerate}
    \item Frekvencija odziva biti će jednaka frekvenciji pobude.
    \item Amplituda odziva biti će skalirana amplituda statičkog pomaka. Prisjetimo
        se da je amplituda statičkog pomaka u izravnoj vezi sa amplitudom sile pobude.
    \item Odziv će zaostajati u fazi za pobudom. 
\end{enumerate}

Skaliranje amplitude statičkog pomaka $(u_{st})_0$ definirano je \textit{dinamičkim
koeficijentom} $R_d$. Navedeni koeficijent je konkretna vrijednost magnitudnog
spektra frekvencijske funkcije odziva za određenu frekvenciju $\omega$. Analogno
tome, kašnjenje u fazi definirano je faznim kutom koji je konkretna vrijednost
faznog spektra frekvencijske funkcije odziva.

\subsubsection{Alternativno zapis prisilnog dijela odziva(Za ovo i nisam baš
siguran)}
Odziv na pobudu sinusnom silom može se odrediti konvolucijom prijenosne funkcije i
funkcije pobude u vremenskoj domeni. Zadana je pobuda sinusnom silom oblika:
\begin{equation}\label{eq:pobuda_sinusnom_silom}
    p(t)=p_0e^{\alpha t} \sin(\omega t)
\end{equation}

Jednadžbu \eqref{eq:pobuda_sinusnom_silom} možemo zapisati u eksponencijalnom
obliku:
\begin{equation}\label{eq:pobuda_sinusnom_silom_eksponencijalni}
    p(t)=-\frac{1}{2}p_0i(e^{(\alpha+i\omega)t}-e^{(\alpha-i\omega)t})
\end{equation}

Uvodimo:
\begin{align}
    s  &= \alpha + i\omega \label{eq:s_normalno}\\
    s^* &= \alpha - i\omega \label{eq:s_konjugirano}
\end{align}
Uvrštavanjem \eqref{eq:s_normalno} i \eqref{eq:s_konjugirano} u
\eqref{eq:pobuda_sinusnom_silom_eksponencijalni} dobijemo:
\begin{equation}\label{eq:pobuda_sinusnom_silom_konacno}
    p(t) = -\frac{1}{2}p_0(e^{st}-e^{s^*t})
\end{equation}

Prijenosna funkcija sustava (s izlučenim članom $1/k$) zadana je u obliku $1/k
h(t)$. Izračun odziva je prikazan u nastavku:
\[
    u(t) = \frac{1}{k}(h*p)(t) 
         = \frac{1}{k}\int_0^\infty h(t-\tau)p(\tau)d\tau
\]

Uvodi se supstitucija $\lambda=t-\tau$.
\[
    u(t) = \frac{1}{k}\int_0^\infty h(\lambda)p(t-\lambda)d\lambda
         = \frac{1}{k}\int_0^\infty h(\lambda)\left[
             -\frac{1}{2}p_0i\left(e^{st-s\lambda}-e^{s^*t-s^*\lambda}\right)
             \right]d\lambda
\]
Raspisivanjem i sređivanjem dobijemo:
\begin{equation}\label{eq:prisilni_transformacija_prijenosne}
    u(t) = -\frac{1}{2}\frac{p_0}{k}i \left[
                e^{st}
                \underbrace{
                    \int_0^\infty h(\lambda)e^{-s\lambda}d\lambda
                }_{\text{\textbf{$I)$}}}
                -e^{s^*t}
                \underbrace{
                    \int_0^\infty h(\lambda)e^{-s^*\lambda}d\lambda
                }_{\text{\textbf{$II)$}}}
        \right]
\end{equation}
Uočimo da integrali označeni s $I)\text{ i } II)$ predstavljaju Laplaceovu
transformaciju prijenosne funkcije sustava pa izraz pod \eqref{eq:prisilni_transformacija_prijenosne}
postaje:
\begin{equation}\label{eq:prisilni_transformacija_prijenosne_kk}
    u(t)=-\frac{1}{2}\frac{p_0}{k}i(e^{st}H(s)-e^{s^*t}H(s))
\end{equation}

Varijabla $s^*$ je kompleksno konjugirana varijabla $s$, pa je funkcija
$H(s^*)$ kompleksno konjugirana funkcija $H(s)$. odnosno:
\begin{equation}\label{eq:prijenosne_odnos}
    H(s^*)=H^*(s) 
\end{equation}

Korištenjem \eqref{eq:prijenosne_odnos} jednadžba \eqref{eq:prisilni_transformacija_prijenosne_kk}
postaje:
\begin{equation}\label{eq:prisilni_transformacija_prijenosne_hk}
    u(t)=-\frac{1}{2}\frac{p_0}{k}i (e^{st}H(s)-e^{s^*t}H^*(s))
\end{equation}

Poseban slučaj je za $\alpha = 0$ odnoso $s=i\omega \text{ i } s^*=-i\omega$. Tada
prijenosne funckije sustava postaju frekvencijske funkcije odziva, a izraz pod
\eqref{eq:prisilni_transformacija_prijenosne_hk} glasi:

\begin{equation}\label{eq:prisilni_transformacija_frf}
    u(t) = -\frac{1}{2}\frac{p_0}{k}i(e^{i\omega t}H(\omega i) - e^{-i\omega t}H^*(i\omega))
\end{equation}

Frekvencijska funkcija odziva $H(\omega i)$ zadana je jednadžbom \eqref{eq:frf_pravokutni}.
Možemo uočiti da je njezin imaginarni dio negativan, što znači da je imaginarni dio
funkcije $H^*(\omega i)$ pozitivan.

Modul funkija $H(\omega)$ i $H^*(\omega)$ je isti a definiran je jednadžbom
\eqref{eq:magnitudniSpektar} a fazni kut jednadžbom \eqref{eq:fazniSpektar}. Sa
slike \ref{fig:frf-gauss} vidljivo je da je kut $\phi$ što ga zatvara $H(\omega)$ s
realnom osi negativan, a kut što ga zatvara $H^*(\omega)$ s realnom osi pozitivan.
Trigonometrijski zapisi funkcija $H(\omega) \text{ i } H^*(\omega)$ dati su u
nastavku.
\begin{align}
    H(\omega) &= |H(\omega)|e^{-i\phi} \label{eq:trig_zapis}\\
    H^*(\omega) &= |H(\omega)|e^{i\phi} \label{eq:trik_zapis_hk}\\ %hk- h-kompleksno konjugiran
\end{align}

Uvrštavanjem \eqref{eq:trig_zapis} i \eqref{eq:trig_zapis_kk} u \eqref{eq:prisilni_transformacija_prijenosne_hk}
dobijemo:
\begin{equation}
    u(t)=-\frac{1}{2}\frac{p_0}{k}i|H(\omega)|(e^{i\omega t}e^{-i\phi}-e^{-i\omega t}e^{i\phi})
        =-\frac{1}{2}\frac{p_0}{k}i|H(\omega)|(2i\sin(\omega t -\phi))
\end{equation}

Te konačno:
\begin{equation}\label{eq:prisilni_alternativno_rjesenje}
    u(t)=\frac{p_0}{k}|H(\omega)|\sin(\omega t - \phi)
\end{equation}
Jednadžba \eqref{eq:prisilni_alternativno_rjesenje} predstavlja prisilni dio odziva.

Definiramo dinamički koeficijent pomaka ili koeficijent povećanja pomaka ($R_d$) kao omjer
amplitude dinamičkog i statičkog pomaka.
\begin{equation}\label{eq:R_d-izvod-konacno}
    R_d = \frac{u_0}{(u_{st})_0}=\frac{1}{\sqrt{(1-(\omega/\omega_n)^2)^2+(2\zeta\omega/\omega_n)^2}}
\end{equation}

Uočimo da su jednadžbe \eqref{eq:R_d-izvod-konacno} i \eqref{eq:frf_sredjeno} iste,
odnosno funkcija ovisnost dinamičkog faktora $R_d$ o omjeru frekvencija
$\omega/\omega_n$ odgovara magnitudnom spektru frekvencijske funkcije odziva. Stoga
jednadžbu pod \eqref{eq:prisilni_alternativno_rjesenje} možemo zapisati kao:
\begin{equation}\label{eq:prisilni_alternativno_rjesenje_Rd}
    u(t)=\frac{p_0}{k}R_d\sin(\omega t - \phi)
\end{equation}

Iz \eqref{eq:prisilni_alternativno_rjesenje} i \eqref{eq:prisilni_alternativno_rjesenje_Rd}
možemo definirati fizikalnu interpretaciju frekvencijske funkcije odziva.
Dakle, frekvencijska funkcija odziva definira odnos između pobude i
odziva. Taj odnos je kompleksan jer je opisan magnitudnim i faznim spektrom
kompleksne funkcije definirane u $s$ domeni. Drugim riječima, frekvencijska funkcija
odziva nas upućuje na slijedeća svojstva (prisilnog) odziva:
\begin{enumerate}
    \item Frekvencija odziva biti će jednaka frekvenciji pobude ($\omega$).
    \item Amplituda odziva biti će skalirana amplituda statičkog pomaka. Prisjetimo
        se da je amplituda statičkog pomaka u izravnoj vezi sa amplitudom sile pobude.
    \item Odziv će zaostajati u fazi za pobudom. 
\end{enumerate}

Skaliranje amplitude statičkog pomaka $(u_{st})_0$ definirano je \textit{dinamičkim
koeficijentom} $R_d$. Navedeni koeficijent je konkretna vrijednost magnitudnog
spektra frekvencijske funkcije odziva za određenu frekvenciju $\omega$. Analogno
tome, kašnjenje u fazi definirano je faznim kutom koji je konkretna vrijednost
faznog spektra frekvencijske funkcije odziva.

\subsection{Grafički prikaz frekvencijske funkcije odziva}
Zbog toga što frekvencijska funkcija odziva definira uvećanje amplitude statičkog
pomaka i fazno zaostajanje odziva za pobudom od posebnog su nam interesa fazni i
magnitudni spektar frekvencijske funkcije odziva (polarni zapis).
\par

Za konstantan $\omega_n$ vrijednosti funkcija definiranih jednadžbama 
\eqref{eq:magnitudniSpektar} i \eqref{eq:fazniSpektar} biti će za omjer
$\omega/\omega_n$, stoga taj omjer možemo postaviti i kao argument navedenih
funkcija. 
\begin{equation}\label{eq:magnitudniSpektarZaOmjer}
    \left|H\left(\frac{\omega}{\omega_n}\right)\right|=
        \frac{1}{\sqrt{((1-(\omega/\omega_n)^2)^2+(2\zeta(\omega/\omega_n))^2}}
\end{equation}

\begin{equation}\label{eq:fazniSpektarZaOmjer}
    \phi\left(\frac{\omega}{\omega_n}\right)=
        \arctan\left(\frac{2\zeta(\omega/\omega_n)}{1-(\omega/\omega_n)}\right)
\end{equation}


Magnitudni spektar frekvencijske funkcije odziva prikazuje funkcijsku ovisnost
dinamičkog koeficijenta $R_d$ o omjeru frekvencije pobude i prirodne frekvencije za
određeno prigušenje. Analogno tome fazni spektar frekvencijske funkcije odziva
prikazuje ovisnost faznog kuta (kašnjenje u fazi za pobudom) o omjeru frekvencija 
$\omega/\omega_n$ za određeni faktor relativnog prigušenja.
\par

Za neprigušeno titranje funkcije faznog i magnitudnog spektra prikazane su u
nastavku.
\begin{equation}\label{eq:fazniSpektarNepriguseno}
    \phi\left(\frac{\omega}{\omega_n}\right)=
        \arctan\left(\frac{0}{1-(\omega/\omega_n)}\right)
\end{equation}
\begin{equation}\label{eq:magnitudniSpektarNepriguseno}
    R_d=\left|H\left(\frac{\omega}{\omega_n}\right)\right|=
        \frac{1}{\sqrt{((1-(\omega/\omega_n)^2)^2+(2\cdot 0(\omega/\omega_n))^2}} =
        \frac{1}{|1-(\omega/\omega_n)^2|}
\end{equation}

Iz formule \eqref{eq:fazniSpektarNepriguseno} očito je da za neprigušeno titranje
postoje tri vrijednosti faznog kuta koje ovise o izrazu u nazivniku. Vrijednosti
faznog kuta prikazane su u nastavku:
\[
    \phi = \begin{dcases*}
            0\degree   & za $1-(\omega/\omega_n)^2 > 0$\\ % & $\omega < \omega_n$\\
            90\degree  & za $1-(\omega/\omega_n)^2 = 0$\\ % & $\omega = \omega_n$\\
            180\degree & za $1-(\omega/\omega_n)^2 < 0$\\ % & $\omega > \omega_n$\\
        \end{dcases*}
\]

Analizirajući krivulje sa \eqref{fig:frf-priguseno} i \eqref{fig:frf-nepriguseno} 
navedene grafove možemo podijeliti na tri područja:
\begin{enumerate}
    \item Područje kontrolirano krustošću - za slučaj spore promjene opterećenja
        ($\omega/\omega_n<<1$ - lijevo na grafu) utjecaj prigušenja je neznatan
        (krivulje za različita prigušenja su jako bliske) a dinamički utjecaj je mali, 
        tj. $R_d \approx 1$ što znači da je amplituda prisilnog odziva približno jednaka
        amplitudi statičkog pomaka, pa amplitudu prisilnog odziva možemo
        aproksimirati slijedećom jednadžbom:
        \begin{equation}\label{eq:frf_prvi_sektor}
            u_0\simeq(u_{st})_0=\frac{p_0}{k}
        \end{equation}
        Fazni kut je približno $0\degree$ pa su pobuda i odziv u fazi.

    \item Područje kontrolirano prigušenjem - Za slučaj $\omega/\omega_n\simeq 1$,
        izražen razmak između krivulja nalaže najeveći utjecaj prigušenja na
        vrijednost dinamičkog faktora, a samim time i na ukupnu amplitudu prisilnog
        odziva. Dinamički faktor $R_d$, u navedenom intervalu, postiže najveće
        vrijednosti a u slučaju $\zeta=0$ $R_d$ je neograničen (teži u beskonačno). 
        Dominantni član izraza \eqref{eq:magnitudniSpektarZaOmjer} je
        $2\zeta(\frac{\omega}{\omega_n})$, a aproksimacija amplitude prisilnog
        odziva glasi: 
        \begin{equation}\label{eq:frf_rezonanca}
            u_0=\frac{(u_{st})_0}{2\zeta}=\frac{p_0}{c\omega_n}
        \end{equation}

        Fazni kut za sva prigušenja iznosi 90\degree. 

    \item Područje kontrolirano masom - Za slučaj brze promjene opterećenja
        $\omega/\omega_n>>1$ utjecaj prigušenja je zanemariv jer su krivulje vrlo
        bliske. Dinamički faktor $R_d$ teži u nulu, što znači da je ukupna amplituda
        prisilnog odziva manja od amplitude statičkog pomaka. Dominantni član jednadžbe 
        \eqref{eq:magnitudniSpektar} jest $(\omega/\omega_n)^4$ što znači
        da jednadžbu pod \eqref{eq:magnitudniSpektarZaOmjer} možemo aproksimirati na 
        slijedeći način:
        \[
            H(\omega/\omega_n)\simeq \frac{\omega_n^2}{\omega^2}
        \]

        Prema tome, amplituda prisilnog odziva glasi:
        \begin{equation}\label{eq:frf_treci_sektor}
            u_0=(u_{st})_0\left(\frac{\omega_n}{\omega}\right)^2=\frac{p_0}{m\omega^2}
        \end{equation}
        
        U ovom slučaju ($\omega/\omega_n >> 1$), fazni kut je približno 180\degree,
        što znači da su pobuda i odziv izvan faze
\end{enumerate}


\subsection{Dinamički koeficijenti odziva}
U prethodnim poglavljima, prisilni dio odziva je prikazan kao vremenska funkcija pomaka
pri čemu je njegova amplituda jednaka statičkom pomaku skaliranom dinamičkim
koeficijentom pomaka $R_d$. Osim vremenskom funkcijom pomaka, odziv sustava potrebno
je opisati i vremenskim funkcijama brzine i ubrzanja.

Deriviranjem vremenske funkcije pomaka po vremenu dobijemo vremensku funkciju brzine:
    \begin{align}
        \frac{\dot{u}(t)}{(u_{st})_0} &= 
            \omega R_d\cos(\omega t - \phi)\frac{\omega_n}{\omega_n}\notag\\
        \frac{\dot{u}(t)}{\omega_n p_0/k} &= 
            \frac{\omega}{\omega_n}R_d\cos(\omega t-\phi)\label{eq:derivacijaFaktorPomaka}\\
        \frac{\dot{u}(t)}{p_0/\sqrt{km}} &=
            R_v\cos(\omega t-\phi)\label{eq:dinamickiFaktorBrzine}
    \end{align}
Gdje je:
\begin{table}[H]
    \begin{tabular}{r l}
        $R_v$ & Dinamički koeficijent brzine\\
    \end{tabular}
\end{table}

Iz jednadžbi pod \eqref{eq:derivacijaFaktorPomaka} i \eqref{eq:dinamickiFaktorBrzine}
slijedi relacija:
\begin{equation}\label{eq:R_v}
    \frac{\dot{u}(0)}{p_0/\sqrt{km}}=R_v=\left(\frac{\omega}{\omega_n}\right)R_d
\end{equation}

Analogno tome, dobije se i dinamički faktor ubrzanja koji glasi:
\begin{equation}\label{eq:R_a}
    \frac{\ddot{u}(t)}{p_0/m}=-R_a=-\frac{\omega}{\omega_n}R_v
        =-\left(\frac{\omega}{\omega_n}\right)^2R_d
\end{equation}

Gdje je:
\begin{table}[H]
    \begin{tabular}{r l}
        $R_a$ & Dinamički koeficijent ubrzanja\\
    \end{tabular}
\end{table}


%Sa slike \ref{fig:rd-rv-ra} vidljivo je da prigušenje na sva tri koeficijenta najviše
%utječe u okolini točke $\omega/\omega_n=1$. Isto tako vidljivo je da je maksimum
%dinamičkog faktora pomaka pomaknut u lijevo od vertikalnog pravca koji prolazi
%točkom $\omega/\omega_n=1$, a maksimum dinamičkog faktora ubrzanja u desno. 
%Maksimum dinamičkog faktora brzine pada na pravac $\omega/\omega_n=1$. Poslijedica
%toga je da se rezonantne frekvencije sustava na pomak, brzinu i ubrzanje razlikuju.
%Rezonantne frekvencije biti će razmatrane detaljnije u slijedećim poglavljima.

Iz \eqref{eq:R_v} i \eqref{eq:R_a} slijedi da su dinamički koeficijenti pomaka,
brzine i ubrzanja u odnosu. Navedeni odnos je prikazan je slijedećom jednadžbom.
\begin{equation}\label{eq:R_d-R_v-R_a}
    \frac{R_a}{\omega/\omega_n}=R_v=\frac{\omega}{\omega_n}R_d 
\end{equation}
Iz navedene jednadžbe slijedi da je poznavanjem jedne od veličina $R_d, R_v 
\text{ ili } R_a$ moguće dobiti preostale dvije. Zbog toga što postoji odnos
između dinamičkih koeficijenata pomaka, brzine i ubrzanja opisan jednadžbom 
\eqref{eq:R_d-R_v-R_a}, moguće je sve tri navedene veličine prikazati u jednom
grafu.

Takav graf se sastoji od četiri logaritamske skale:
\begin{enumerate}
    \item horizontalne logaritamske skale koja prikazuje omjer frekvencije pobude i 
        prirodne frekvencije sustava 
    \item vertikalne logaritamske skale koja prikazuje \textit{dinamički koeficijent brzine
        $R_d$}
    \item modificirane logaritamske skale nagnute pod kutem od $+45\degree$ koja prikazuje
        \textit{dinamički koeficijent ubrzanja $R_a$}
    \item modificirane logaritamske skale nagnute pod kutem od $-45\degree$ koja prikazuje
        \textit{dinamički koeficijent pomaka $R_d$}
\end{enumerate}


\section{Rezonanca}
Rezonanca je pojava koja se javlja prilikom pobude rezonantnom frekvencijom.
Rezonantna frekvencija je ona frekvencija pobude za koju će dinamički koeficijent
odziva biti maksimalan. 
\par

\subsection{Rezonanca sustava s prigušenjem}
Razmatranjem krivulja frekvencijskih funkcija odziva sa slike 
\ref{fig:rd-rv-ra}, možemo uočiti da jedino maksimumi krivulja $R_v$ padaju na 
vertikalni pravac $\omega/\omega_n=1$. Isto tako, vidljivo je da maksimalni $R_d$
pada ulijevo od navedenog pravca a maksimalni $R_a$ udesno. To znači da će se 
rezonantne frekvencije dinamičkih faktora ($R_d$, $R_v$ i $R_a$) međusobno razlikovati 
te da će rezonantne frekvencije dinamičkih faktora $R_d$ i $R_a$ biti različite od
prirodne frekvencije sustava.

Rezonantne frekvencije za pojedini spektar možemo odrediti deriviranjem frekvencijske 
funkcije odziva po $\omega/\omega_n$ te izjednačavanjem prve derivacije s
nulom.
\[
    R_d = \frac{1}{\sqrt{(1-(\omega/\omega_n)^2) + (2\zeta(\omega/\omega_n))^2}}
\]
Uvodi se supstitucija $x=\omega/\omega_n$
\[
    R_d = \frac{1}{\sqrt{(1-x^2)+(2\zeta x)^2}}
\]
Deriviranjem po x dobijemo:
\[
    \frac{dR_d}{dx}=-\frac{2x^3+(4\zeta^2-2)x}{\sqrt{((1-x^2)^2+(2\zeta x)^2)^3}}
\]
Lokalni ekstrem (maksimum) dobijemo izjednačavanjem prve derivacije s nulom,
odnosno:
\[
    -\frac{2x^3+(4\zeta^2-2)x}{\sqrt{((1-x^2)^2+(2\zeta x)^2)^3}} = 0
\]
Da bi razlomak bio jednak nula izraz u brojniku mora biti jednak nula.
Izjednačavanjem brojnika s nulom dobijemo slijedeću jednadžbu:
\[
    \begin{aligned}
        2x^3+(4\zeta^2-2)x &= 0\\
        x^2 &= 1-2\zeta^2\\
        x &= \sqrt{1-2\zeta^2} \quad \text{ Uz } \zeta < \frac{1}{\sqrt{2}}
    \end{aligned}
\]
Uvrštavanjem $\omega/\omega_n$ dobijemo izraz za rezonantnu frekvenciju pomaka:
\begin{equation}\label{eq:rezonantna_frekvencija_pomak}
    \omega=\omega_n\sqrt{1-2\zeta^2}
\end{equation}
Analogno tome dobiju se izrazi za rezonantne frekvencije brzine i ubrzanja.
\begin{align}
    \text{Rezonantna frekvencija pomaka}\quad & \omega = \omega_n\sqrt{1-\zeta^2}\label{eq:rd_rezonanca}\\
    \text{Rezonantna frekvencija brzine}\quad & \omega = \omega_n\label{eq:rv_rezonanca}\\
    \text{Rezonantna frekvencija ubrzanja}\quad & \omega = \frac{\omega_n}{\sqrt{1-2\zeta^2}}\label{eq:ra_rezonanca}
\end{align}

Zbog jednostavnosti, karakteristike odziva prigušenog sustava s jednim stupnjem
slobode u rezonanci, razmotriti ćemo za slučaj $\omega = \omega_n$ uz
homogene početne uvjete. Iako se rezonantna frekvencija pomaka ne podudara s 
prirodnom frekvencijom sustava, navedene vrijednosti su vrlo bliske za mali $\zeta$. 
Vremenska funkcija pomaka prigušenog sustava prikazana jednadžbom \eqref{eq:inverz_treci_korak}

Uz homogene početne uvjete i za $\omega=\omega_n$, koeficijenti $A,B,C,\text{ i }D$
glase:
\begin{align}
    C &= 0 \label{eq:rez_koef_C}\\
    D &= -\frac{(u_{st})_0}{2\zeta}\label{eq:rez_koef_D}\\
    A &= -D = \frac{(u_{st})_0}{2\zeta}\label{eq:rez_koef_A}\\
    B &= -\frac{D\sigma}{\omega_D}-\frac{C\omega}{\omega_D}=
          \frac{(u_{st})_0}{2\sqrt{1-\zeta^2}}\label{eq:rez_koef_B}
\end{align}
Uvrštavanjem \eqref{eq:rez_koef_A}, \eqref{eq:rez_koef_B}, \eqref{eq:rez_koef_C} i \eqref{eq:koef_D}
u \eqref{eq:inverz_treci_korak} te sređivanjem dobijemo:
\begin{equation}\label{eq:rezonancaPrigušenoPomak}
    u(t)=\frac{(u_{st})_0}{2\zeta}\left[
        e^{-\sigma t} \left(
            \cos(\omega_D t)+\frac{\zeta}{\sqrt{1-\zeta^2}}\sin(\omega_D t)
            \right)
        -\cos(\omega_n t)
        \right]
\end{equation}
Iz navedene jednadžbe proizlazi da je amplituda odziva ograničena na vrijednost
(uočimo da je odziv sustava kontroliran prigušenjem):
\begin{equation}\label{eq:rezonanca_amplituda}
    u_0=\frac{(u_{st})_0}{2\zeta}
\end{equation}
Za malu vrijednost $\zeta$, vrijedi $\omega_n\approx\omega_D$ te je član
$\sin(\omega_Dt) \approx 0$. Stoga jednadžba pod \eqref{eq:rezonancaPrigušenoPomak}
postaje:
\begin{equation}\label{eq:rezonancaOdzivAproksimacija}
    u(t)\simeq\underbrace{
        \frac{(u_{st})_0}{2\zeta}[(e^{-\sigma t}-1)
        }_{\text{Krivulja ovojnice}}
        \cos(\omega_nt)]
\end{equation}

Iz jednadžbe \eqref{eq:rezonancaOdzivAproksimacija} može se zaključiti da i za
slučaj rezonance postoji prolazni i prisilni dio odziva, a ukupni odziv je razlika
između prolaznog i prisilnog djela. 
Prolazni dio odziva opisan je jednadžbom
\[
    e^{-\sigma t} \cos(\omega_n t)
\]
a prisilni:
\[
    \cos(\omega_n t)
\]
Za $t=0$ prolazni dio je maksimalan te je ukupni odziv jednak nuli. U ovisnosti o
vremenu, prolazni dio se smanjuje eksponencijalno prema zakonu $e^{-\sigma t}$. Kako
se prolazni dio smanjuje a prisilni ostaje isti ($(u_{st})_0/2\zeta$) tako raste
njihova razlika. Rastom razlike dolazi do rasta amplitude odziva, te isčezavanjem
prolaznog dijela preostaje samo prisilni te se dostiže maksimalna amplituda koja je
jednaka ($(u_{st})_0/2\zeta$). Bitno je za naglasiti da u teoretskom modelu
prolazni dio odziva isčezava tek za $t=\infty$, tj. asimptotski se približava nuli,
ali u realnosti prolazni dio odziva je zanemariv nakon određenog vremena. Shematski
prikaz dostizanja prisilnog stanja prikazan je na slijedećoj slici:

O prigušenju u rezonanci ovise slijedeći parametari odziva:
\begin{itemize}
    \item brzina dostizanje ustaljenog stanja (maksimalne amplitude) - brzina dostizanja 
        ustaljenog stanja raste proporcionalno s prigušenjem (veće prigušenje $\to$ strmija krivulja
        ovojnice $\to$ brže dostizanje ustaljenog stanja).

    \item vrijednost maksimalne amplitude - obrnuto proporcionalna od vrijednosti
        prigušenja, definirana izrazom \eqref{eq:rezonanca_amplituda}. 
        (veće prigušenje, manja maksimalna amplituda, vidljivo i u frekvencijskim funkcijama odziva)
\end{itemize}

Određivanje broja titraja koji je potreban za dostizanje ustaljenog stanja vrši
se pomoću funkcije koja opisuje krivulju ovojnice. Pretpostavka je da ekstrem
nastupa nakon $j$ titraja ($j$ je prirodni broj), a vrijeme nastupa minimuma je
$t=2\pi j/\omega$. 
\begin{equation}\label{eq:prirastEkstremaPriguseno}
    u\left(\frac{2\pi j}{\omega_n}\right) \approx
        u_0(e^{-\zeta\omega_n\frac{2\pi j}{\omega_n}}-1)\cos\left(\omega_n\frac{2\pi
        j}{\omega_n}\right)
\end{equation}
Gdje je:
\begin{table}[H]
    \begin{tabular}{c c}
        $j$ & redni broj titraja\\
        $u_0=(u_{st})_0/2\zeta$ & maksimalna amplituda\\
    \end{tabular}
\end{table}
Kako se radi o ekstremnoj vrijednosti, funkcija kosinus iznosi $\pm 1$ pa jednadžba
pod \eqref{eq:prirastEkstremaPriguseno} glasi:
\begin{equation}\label{eq:MiniMaxPriguseno}
     u\left(\frac{2\pi j}{\omega_n}\right) = u_j = 
        \pm u_0(e^{-2\pi\zeta j}-1)
\end{equation}

Za maksimume jednadžba pod \eqref{eq:MiniMaxPriguseno} postaje
\begin{equation}
    |u_j| = -u_0(e^{-2\pi\zeta j} -1) = u_0(1-e^{-2\pi\zeta j})
\end{equation}

Za relativne vrijednosti\footnote{Postotci od maksimalne amplitude}:
\begin{equation}
    u[j] = \frac{|u_j|}{u_0}=1-e^{-2j\zeta\pi}
\end{equation}
Izraz ima smisla samo za diskretne vrijednosti argumenta $j$, odnosno za $j \in
\mathbb{N}$.

\par
\begin{table}[H]
    \begin{tabular}{c | c}
       \hline
        $\zeta$ & $j$\\
        \hline
        $0.01$ & $48$\\
        \hline
        $0.02$ & $24$\\
        \hline
        $0.05$ & $10$\\
        \hline
        $0.1$ & $5$\\
        \hline
        $0.2$ & $3$\\
        \hline
    \end{tabular}
    \caption{Očitanje s grafa}
    \label{table:prirast-rezonanca-priguseno}
\end{table}

Uočimo da je uz slabije prigušenje potrebno više titraja za dostizanje ustaljenog
stanja odnosno maksimalne amplitude. Očitanja vrijednosti sa grafa prikazana su u
\ref{table:prirast-rezonanca-priguseno}.

\subsection{Rezonanca sustava bez prigušenja}
Za sustav bez prigušenja, rezonantne frekvencije za $R_d$, $R_v$ i $R_a$ jednake su
prirodnoj frekvenciji sustava što se dobije uvrštavanjem $\zeta = 0$ u \eqref{eq:rd_rezonanca} i 
\eqref{eq:ra_rezonanca}. 
\par

Primjetimo da je maksimalni dinamički koeficijent $R_d$ (za $\omega/\omega_n = 1$) 
neograničen, tj $R_d\to \infty$ što se vidi i u jednadžbi \eqref{eq:fazniSpektarNepriguseno} 
te na grafu \ref{fig:frf-nepriguseno}. U slijedećoj jednadžbi prikazana je vremenska 
funkcija pomaka sustava za homogene početne uvjete:
\begin{equation}
    u(t)=\frac{p_0}{k}\frac{1}{1-(\omega/\omega_n)}
            \left(\sin(\omega t) - \frac{\omega}{\omega_n}\sin(\omega_n t)\right)
\end{equation}

Uočimo da za $\omega = \omega_n$ navedena jednadžba više ne vrijedi (djeljenje s
nulom). Novu jednadžbu možemo odrediti na slijedeći način:
\begin{equation}\label{eq:limes}
    \lim_{\omega\to\omega_n}{u(t)} = 
        \frac{p}{k}\frac{1}{1-(\omega/\omega_n)^2}
            \left(\sin(\omega t) - \frac{\omega}{\omega_n}\sin(\omega_n t)\right)
\end{equation}

Navedeni limes je oblika $\frac{0}{0}$, pa ga je moguće rješiti L'Hopitalovim
pravilom. Deriviranjem funkcije po $\omega$ dobijemo:
\begin{equation}\label{eq:lhopitalovo_limes}
    \lim_{\omega\to\omega_n}\frac{d}{d\omega}u(t)=
    \lim_{\omega\to\omega_n} \left[
        \frac{p_0}{k}\frac{1}{-2(\omega/\omega_n)}
            \left(t\cos(\omega t) - \frac{1}{\omega}\sin(\omega_n t)\right)
           \right]
\end{equation}
Uvrštavanjem $\omega=\omega_n$ dobijemo:
\begin{equation}\label{eq:odziv_rezonanca}
    u(t)=-\frac{1}{2}\frac{p_0}{k}(\omega_n\cos(\omega_n t)-\sin(\omega_nt))
\end{equation}

Iz navedene jednadžbe vidljivo je da usprkos neograničenom dinamičkom faktoru
neizmjerno velika amplituda ne nastupa trenutno, već dolazi do njezinog 
postupnog rasta. Djeljenjem izraza \eqref{eq:odziv_rezonanca} statičkim pomakom i
uvrštavanjem $\omega_n=\frac{2\pi}{T_n}$ dobijemo:
\begin{equation}\label{eq:rezonanca_period}
    \frac{u(t)}{(u_{st})_0}=-\frac{1}{2}
        \left(\frac{2\pi t}{T_n}
                \cos\left(\frac{2\pi t}{T_n}\right)
                -
                \sin\left(\frac{2\pi t}{T_n}\right)
        \right)
\end{equation}
Gdje je:
\begin{table}[H]
    \begin{tabular} {r l}
        $T_n$ & period titranja\\
    \end{tabular}
\end{table}

Iz prethodne jednadžbe slijedi da ekstremi nastupaju svaki poluperiod ($T_n/2$), pri
čemu prvo nastupa maksimum a zatim minimum. Vrijeme nastupa ekstrema za određeni
redni broj titraja prikazuju slijedeće jednadžbe:
\begin{itemize}
    \item za maksimum: $t=(i-\frac{1}{2})T_n$
    \item za minimum: $t=jT_n$
\end{itemize}
Gdje je:
\begin{table}[H]
    \begin{tabular} {r l}
        $t$ & vrijeme nastupa ekstrema\\
        $i$ & redni broj titraja\\
    \end{tabular}
\end{table}
Iznos ekstrema određujemo uvrštavanjem vremena nastupa ekstrema u jednadžbu
\eqref{eq:rezonanca_period} te slijedi:
\begin{enumerate}
    \item iznos maksimuma za j-ti titraj: $u_j=(u_{st})_0\pi(j-\frac{1}{2})$
    \item iznos minimuma za j-ti titraj: $u_j=-(u_{st})_0\pi\cdot j$ 
\end{enumerate}

Prirast maksimuma određujemo razlikom između iznosa maksimuma trenutnog i slijedećeg
titraja što prikazuje slijedeća jednadžba:
\begin{equation}\label{eq:prirast_maksimuma}
    \begin{split}
        |u_{j+1}|-|u_j| &=(u_{st})_0\pi((j+1)-\frac{1}{2})-(u_{st})_0\pi(j-\frac{1}{2})\\
        |u_{j+1}|-|u_j| &= \frac{p_0}{k}\pi
    \end{split}
\end{equation}

Analogno tome određuje se i prirast minimuma koji glasi:
\begin{equation}\label{eq:prirast_minimuma}
    \begin{split}
        u_{j+1}-u_j &= -(u_{st})_0\pi(j+1)-(-(u_{st})_0\pi j)\\
        u_{j+1}-u_j &= -\frac{p_0}{k}\pi
    \end{split}
\end{equation}

Uočimo da su prirasti ekstrema linearni, stoga krivulju ovojnice čine pravaci čiji
su koeficjenti smjera prikazani u nastavku:
\begin{equation}\label{eq:koef_smjera_envelopa}
    k_{1,2}=\pm\frac{1}{2}\frac{p_0}{k}\omega_n
\end{equation}

\subsection{Analiza područja rezonance}
Osim rezonantne frekvencije potrebno je odrediti i pojas polovice snage vrha
frekvencijske funkcije odziva, koji je prikazan na slici \ref{fig:hpb}. Pojas polovice
snage vrha frekvencijske funkcije odziva bitan je iz dva razloga:
\begin{enumerate}
    \item osim pobude rezonantnom frekvencijom, opasne su i pobude frekvencijama iz
        njezinog okoliša. Navedeni okoliš definiran je pojasom polovice snage.
    \item Zbog praktične primjene - pojas polovice snage koristi se u pokusima
        za određivanje stupnja prigušenja konstrukcija
\end{enumerate}

Dinamički faktor pomaka $R_d$ za odziv dvostruko manje snage od odziva maksimalnog
dinamičkog faktora računa se prema slijedećoj relaciji:
\begin{equation}\label{eq:hpb}
    R_d = \frac{1}{\sqrt{2}}R_d^{max}
\end{equation}

Sa slike je vidljivo da je dinamički faktor $R_d$ iz \eqref{eq:hpb} definiran za
dvije vrijednosti frekvencije pobude: $\omega_a$ i $\omega_b$. Raspisivanjem
jednadžbe pod \eqref{eq:hpb} dobijemo:

\begin{equation}\label{eq:hpb_izvod_1}
        \frac{1}{\sqrt{(1-(\omega/\omega_n)^2)^2+(2\zeta\omega/\omega_n)^2}}=
            \frac{1}{\sqrt{2}}\frac{1}{2\zeta\sqrt{1-\zeta^2}}\\
\end{equation}

Kvadriranjem \eqref{eq:hpb_izvod_1} dobijemo:
\begin{equation}\label{eq:hpb_izvod_2}
    \left(1-\left(\frac{\omega}{\omega_n}\right)^2\right)^2
    +\left(2\zeta\frac{\omega}{\omega_n}\right)^2 =
    8\zeta^2(1-\zeta^2)
\end{equation}

Raspisivanjem i grupiranjem po $\omega/\omega_n$ dobijemo:
\begin{equation}\label{eq:hpb_izvod_3}
    \left(\frac{\omega}{\omega_n}\right)^4
    -2(1-2\zeta^2)\left(\frac{\omega}{\omega_n}\right)^2
    +1-8\zeta^2(1-\zeta^2)=0
\end{equation}

Izraz \eqref{eq:hpb_izvod_3} je kvadratna jednadžba, a njezinim rješavanjem (po
$\omega/\omega_n$) dobijemo:
\begin{equation}\label{eq:hpb_kvadratna_1}
    \left(\frac{\omega}{\omega_n}\right)^2 = 
        (1-2\zeta^2)\pm 2\zeta\sqrt{1-\zeta^2}
\end{equation}

Za $\zeta^2 \approx 0$ izraz \eqref{eq:hpb_kvadratna_1} postaje:
\begin{equation}
    \left(\frac{\omega}{\omega_n}\right)^2 \approx
        1 \pm 2\zeta
\end{equation}

Odnosno:
\begin{equation}\label{eq:hpb_kvadratna_2}
    \left(\frac{\omega}{\omega_n}\right)\approx
        \sqrt{1 \pm 2\zeta}
\end{equation}
Jednadžba \eqref{hpb_kvadratna_2}, nakon aproksimacije korijena s prva dva člana Taylorovog
reda glasi:
\begin{equation}\label{eq:hpb_kvadratna_konacno}
    \frac{\omega}{\omega_n} = 1 \pm \zeta
\end{equation}

Frekvencije $\omega_a$ i $\omega_b$ dobiju se iz \eqref{eq:hpb_kvadratna_konacno}:
\begin{align}
    \omega_a &= (1-\zeta)\omega_n\\
    \omega_b &= (1+\zeta)\omega_n\\
\end{align}

Oduzimanjem $\omega_b-\omega_a$ dobijemo:
\begin{equation}\label{eq:hpb_zeta_omega}
        2\zeta = \frac{\omega_b-\omega_a}{\omega_n}
\end{equation}

Te konačno, dijeljenjem brojnika i nazivnika s $2\pi$:
\begin{equation}\label{eq:hpb_zeta_f}
    \zeta\approx\frac{f_b-f_a}{2f_n}
\end{equation}

Gdje je $f=\omega/2\pi$ kružna frekvencija. Jednadžbe pod \eqref{eq:hpb_zeta_omega}
i \eqref{eq:hpb_zeta_f} bitne su jer omogućuju određivanje koeficijenta relativnog prigušenja
$\zeta$ bez potrebe za poznavanjem intenziteta sile pobude. 

\subsection{Praktična primjena matematičkog modela}
Definiranjem matematičkog modela prigušenog sustava s jednim stupnjem slobode,
pobuđenog sinusnom silom postavljeni su temelji za eksperimentalno određivanje
stupnja prigušenja i prirodne frekvencije. Stupnanj prigušenja jest veličina od
izuzetne praktične važnosti, a nije ga moguće odrediti teoretski iz projektnih
parametara, već se mora odrediti eksperimentalno. %Isto tako izmjerena vrijednost prirodne
%frekvencije jest stvarno svojstvo konstrukcije, s kojim se uspoređuje teoretski
%određena prirodna frekvencija. 
Ispitivanje koje će biti razmatrano u ovome radu naziva se \textit{rezonancijski
pokus}.

%Matematički model prigušenog sustava s jednim stupnjem slobode, pobuđen sinusnom
%silom koristi se priklikom određivanja stupnja prigušenja i prirodne frekvencije 
%realnih konstrukcija. Stupanj prigušenja je veličina od izuzetne važnosti koju nije
%moguće odrediti analitički, nego ju je potrebno odrediti ispitivanjima. 

Ispitivanja se provode \textit{vibracijskim uređajem}. Vibracijski uređaj se sastoji
od dvije košare sa utezima na uspravnoj osovini koje rotiraju u suprotnim smjerovima
konstantnom kutnom brzinom $\omega$. Osovina je pričvršćena za metalnu ploču koja se
kruto povezuje s građevinom. 

Sila pobude građevine jest ukupna centrifugalna sila vibracijskog uređaja, koja je 
jednaka je sumi centrifugalnih sila pojedinih masa.
Horizontalne komponente su jednakog intenziteta ali suprotnog smjera pa se
poništavaju, stoga sila pobude je jednaka sumi vertikalnih komponenti, odnosno:
\begin{equation}
    p(t)=(m_ee\omega^2)\sin(\omega t)
\end{equation}

Odziv sustava s jednim stupnjem slobode na pobudu vibracijskim uređajem opisan je
slijedećom diferencijalnom jednadžbom:
\begin{equation}
    m\ddot{u}+c\dot{u}+ku=(m_ee\omega^2)\sin(\omega t)
\end{equation}

Amplituda prisilnog pomaka glasi (iz \eqref{eq:prisilni_alternativno_rjesenje_Rd}):
\begin{equation}
    u_0=\frac{m_ee}{k}\omega^2R_d = \frac{m_ee}{m}\left(\frac{\omega}{\omega_n}\right)^2R_d
\end{equation}

Amplituda prisilnog ubrzanja (iz \eqref{eq:R_a}
\begin{equation}
    \ddot{u}_0=\frac{m_ee}{m}\omega^2R_a=\frac{m_ee\omega_n^2}{m}\left(\frac{\omega}{\omega_n}\right)^2R_a
\end{equation}


Na grafu sa slike \ref{fig:ra-vibracijski} vidljivo je da daljnjim porastom frekvencije pobude (iznad prirodne
frekvencije) amplituda prisilnog ubrzanja raste. Navedeni rast se događa jer je
amplituda pobude proporcionalna s $\omega^2$. 
\par

Za određivanje stupnja prigušenja i prirodne frekvencije vrši se
\textit{rezonancijski pokus}, a temelji se na slijedećoj relaciji (iz \eqref{eq:frf_rezonanca})
\begin{equation}\label{eq:rezonancijski_pokus}
    \zeta = \frac{1}{2}\frac{(u_{st})_0}{(u_0)_{\omega=\omega_n}}
\end{equation}
Potrebno je eksperimentalno odrediti amplitudu statičkog pomaka i prirodnu frekvenciju. 
%U eksperimentima se redovito mjeri amplituda ubrzanja a amplitudu pomaka možemo
%dobiti slijedećom relacijom:
%\begin{equation}
%    u_0=\frac{\ddot{u}_0}{\omega^2}
%\end{equation}

Prirodna frekvencija se određuje na slijedeći način:
\begin{enumerate}
    \item pobuđivanje konstrukcije vibracijskim uređajem namještenim na
        određenu frekvenciju $\omega$.
    \item očitavanje faznog kuta. Ako je fazni kut $\phi = 90\degree$, tada je
        prirodna frekvencija $\omega_n$ jednaka frekvenciji pobude $\omega$.
    \item nakon što isčezne prolazni dio odziva, očitava se amplituda prisilnog ubrzanja
\end{enumerate}

Amplitudu prisilnog pomaka možemo dobiti iz amplitude prisilnog ubrzanja korištenjem
slijedeće formule:
\begin{equation}
    u_0=\frac{(\ddot{u}_0)_{\omega=\omega_n}}{(\omega^2)_{\omega=\omega_n}} \quad
    \text{ jer je } \quad
    \ddot{u}(t) = -u_0\omega\sin(\omega t -\phi), \quad
\end{equation}

Da bi bilo moguće odrediti prigušenje sustava prema formuli \eqref{eq:rezonancijski_pokus} 
potrebno je odrediti amplitudu statičkog pomaka $(u_{st})_0=p_{0,max}/k$, gdje je
$p_{0,max}$ amplituda pobude u rezonanci. Amplituda statičkog pomaka se
\textbf{mora} odrediti pokusom, a ne izračunati prema relaciji $p_0/k$ zato što $k$
nije eksperimentalno određen.Vibracijskim uređajem je vrlo teško (ili
nemoguće) prouzročiti veliku \textbf{statičku} silu pobude. Dva su pristupa rješavanju
navedenog problema:
\begin{enumerate}
    \item Sporim rotiranjem velikih masa - Nije najbolje rješenje jer je sila pobude
        proporcionalna s kvadratom kutne brzine rotacije utega. Stoga i za velike mase 
        utega amplituda sile pobude je relativno mala.
    \item Povlačenjem konstrukcije užetom silom koja je jednaka amplitudi sile
        pobude vibracijskim uređajem $p_{0,max}$.
\end{enumerate}

Osim rezonancijskim pokusom, prigušenje i prirodnu frekvenciju moguće je
odrediti \textbf{frekvencijskim krivuljama odziva}. Postupak je slijedeći:
\begin{enumerate}
    \item pobuđivanje konstrukcije vibracijskim uređajem namještenim na određenu
        frekvenciju
    \item određivanje amplitude \textbf{prisilnog} dijela 
    \item namještanje vibracijskog uređaja na drugu frekvenciju, te ponavljanje
        postupka%\footnote{Frekvencije vibracijskog uređaja nalaze se u širem spektru 
%        frekvencija, koji uključuje rezonantnu frekvenciju i frekvencije iz njezine 
%        okoline (i lijevo i desno).} 
\end{enumerate}

Frekvencijska krivulja odziva iscrtava se iz izmjerenih podataka. Frekvencijske
funkcije odziva mogu prikazivati slijedeće ovisnosti:
\begin{enumerate}
    \item ovisnost amplituda ubrzanja o frekvencijskom omjeru - izravno iz
        izmjerenih podataka. Bitno je za naglasiti da je navedena krivulja
        proporcionalna s $\omega^2$.

    \item ovisnost dinamičkog faktora ubrzanja o frekvencijskom omjeru (konstantna
        amplituda pobude) - dijeljenjem izmjerenih podataka s $\omega^2$ 

    \item ovisnost dinamičkog faktora pomaka o frekvencijskom omjeru (konstantna
        amplituda pobude) - dijeljenjem izmjerenih podataka s $\omega^4$.
\end{enumerate}

Stupanj prigušenja i prirodna frekvencija može se odrediti iz bilo koje od navedenih
krivulja. Prirodna frekvencija jednaka je frekvenciji sile pobude u rezonanci.
Stupanj prigušenja određuje se iz \textit{pojasa polovice snage} jednadžbom
\eqref{eq:hpb_zeta_f}, što znači da je potrebno odrediti amplitudu odziva pri
rezonanci te frekvencije za koje je amplituda odziva jednaka $r_{res}/\sqrt{2}$.

