\subsection{Grafički prikaz frekvencijske funkcije odziva}
Zbog toga što frekvencijska funkcija odziva definira uvećanje amplitude statičkog
pomaka i fazno zaostajanje odziva za pobudom od posebnog su nam interesa fazni i
magnitudni spektar frekvencijske funkcije odziva (polarni zapis).
\par

Za konstantan $\omega_n$ vrijednosti funkcija definiranih jednadžbama 
\eqref{eq:magnitudniSpektar} i \eqref{eq:fazniSpektar} biti će za omjer
$\omega/\omega_n$, stoga taj omjer možemo postaviti i kao argument navedenih
funkcija. 
\begin{equation}\label{eq:magnitudniSpektarZaOmjer}
    \left|H\left(\frac{\omega}{\omega_n}\right)\right|=
        \frac{1}{\sqrt{((1-(\omega/\omega_n)^2)^2+(2\zeta(\omega/\omega_n))^2}}
\end{equation}

\begin{equation}\label{eq:fazniSpektarZaOmjer}
    \phi\left(\frac{\omega}{\omega_n}\right)=
        \arctan\left(\frac{2\zeta(\omega/\omega_n)}{1-(\omega/\omega_n)}\right)
\end{equation}


Magnitudni spektar frekvencijske funkcije odziva prikazuje funkcijsku ovisnost
dinamičkog koeficijenta $R_d$ o omjeru frekvencije pobude i prirodne frekvencije za
određeno prigušenje. Analogno tome fazni spektar frekvencijske funkcije odziva
prikazuje ovisnost faznog kuta (kašnjenje u fazi za pobudom) o omjeru frekvencija 
$\omega/\omega_n$ za određeni faktor relativnog prigušenja.
\par

Za neprigušeno titranje funkcije faznog i magnitudnog spektra prikazane su u
nastavku.
\begin{equation}\label{eq:fazniSpektarNepriguseno}
    \phi\left(\frac{\omega}{\omega_n}\right)=
        \arctan\left(\frac{0}{1-(\omega/\omega_n)}\right)
\end{equation}
\begin{equation}\label{eq:magnitudniSpektarNepriguseno}
    R_d=\left|H\left(\frac{\omega}{\omega_n}\right)\right|=
        \frac{1}{\sqrt{((1-(\omega/\omega_n)^2)^2+(2\cdot 0(\omega/\omega_n))^2}} =
        \frac{1}{|1-(\omega/\omega_n)^2|}
\end{equation}

Iz formule \eqref{eq:fazniSpektarNepriguseno} očito je da za neprigušeno titranje
postoje tri vrijednosti faznog kuta koje ovise o izrazu u nazivniku. Vrijednosti
faznog kuta prikazane su u nastavku:
\[
    \phi = \begin{dcases*}
            0\degree   & za $1-(\omega/\omega_n)^2 > 0$\\ % & $\omega < \omega_n$\\
            90\degree  & za $1-(\omega/\omega_n)^2 = 0$\\ % & $\omega = \omega_n$\\
            180\degree & za $1-(\omega/\omega_n)^2 < 0$\\ % & $\omega > \omega_n$\\
        \end{dcases*}
\]

Analizirajući krivulje sa \eqref{fig:frf-priguseno} i \eqref{fig:frf-nepriguseno} 
navedene grafove možemo podijeliti na tri područja:
\begin{enumerate}
    \item Područje kontrolirano krustošću - za slučaj spore promjene opterećenja
        ($\omega/\omega_n<<1$ - lijevo na grafu) utjecaj prigušenja je neznatan
        (krivulje za različita prigušenja su jako bliske) a dinamički utjecaj je mali, 
        tj. $R_d \approx 1$ što znači da je amplituda prisilnog odziva približno jednaka
        amplitudi statičkog pomaka, pa amplitudu prisilnog odziva možemo
        aproksimirati slijedećom jednadžbom:
        \begin{equation}\label{eq:frf_prvi_sektor}
            u_0\simeq(u_{st})_0=\frac{p_0}{k}
        \end{equation}
        Fazni kut je približno $0\degree$ pa su pobuda i odziv u fazi.

    \item Područje kontrolirano prigušenjem - Za slučaj $\omega/\omega_n\simeq 1$,
        izražen razmak između krivulja nalaže najeveći utjecaj prigušenja na
        vrijednost dinamičkog faktora, a samim time i na ukupnu amplitudu prisilnog
        odziva. Dinamički faktor $R_d$, u navedenom intervalu, postiže najveće
        vrijednosti a u slučaju $\zeta=0$ $R_d$ je neograničen (teži u beskonačno). 
        Dominantni član izraza \eqref{eq:magnitudniSpektarZaOmjer} je
        $2\zeta(\frac{\omega}{\omega_n})$, a aproksimacija amplitude prisilnog
        odziva glasi: 
        \begin{equation}\label{eq:frf_rezonanca}
            u_0=\frac{(u_{st})_0}{2\zeta}=\frac{p_0}{c\omega_n}
        \end{equation}

        Fazni kut za sva prigušenja iznosi 90\degree. 

    \item Područje kontrolirano masom - Za slučaj brze promjene opterećenja
        $\omega/\omega_n>>1$ utjecaj prigušenja je zanemariv jer su krivulje vrlo
        bliske. Dinamički faktor $R_d$ teži u nulu, što znači da je ukupna amplituda
        prisilnog odziva manja od amplitude statičkog pomaka. Dominantni član jednadžbe 
        \eqref{eq:magnitudniSpektar} jest $(\omega/\omega_n)^4$ što znači
        da jednadžbu pod \eqref{eq:magnitudniSpektarZaOmjer} možemo aproksimirati na 
        slijedeći način:
        \[
            H(\omega/\omega_n)\simeq \frac{\omega_n^2}{\omega^2}
        \]

        Prema tome, amplituda prisilnog odziva glasi:
        \begin{equation}\label{eq:frf_treci_sektor}
            u_0=(u_{st})_0\left(\frac{\omega_n}{\omega}\right)^2=\frac{p_0}{m\omega^2}
        \end{equation}
        
        U ovom slučaju ($\omega/\omega_n >> 1$), fazni kut je približno 180\degree,
        što znači da su pobuda i odziv izvan faze
\end{enumerate}



